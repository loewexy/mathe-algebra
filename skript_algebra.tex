% !TeX spellcheck = de_DE
\documentclass[12pt,parskip=full]{scrartcl}
 
\usepackage[utf8]{inputenc}
\usepackage[T1]{fontenc}
\usepackage{lmodern}
\usepackage[ngerman]{babel}
\usepackage{color}
\usepackage{amsmath,amssymb,amstext,mathtools,amsthm}
\usepackage{subcaption}
\usepackage{float}
\usepackage{wasysym}
\usepackage{stmaryrd}
\usepackage{bbm}
\usepackage[hidelinks]{hyperref}
\hypersetup{bookmarksnumbered}

\usepackage{tikz}
\usetikzlibrary{positioning}
\usetikzlibrary{arrows}

\usepackage{enumitem}
\setenumerate{label=\arabic*)}

\newcommand{\setN}{\mathbb{N}}
\newcommand{\setZ}{\mathbb{Z}}
\newcommand{\setQ}{\mathbb{Q}}
\newcommand{\setR}{\mathbb{R}}
\newcommand{\setC}{\mathbb{C}}
\newcommand{\setH}{\mathbb{H}}
\newcommand{\setk}{\Bbbk}
\newcommand{\ldot}{\,.\,}
\newcommand{\Forall}{~\forall}
\newcommand{\Exists}{~\exists}

\newcommand{\abs}[1]{{\left| #1 \right|}}
\newcommand{\dabs}[1]{{\left\lVert #1 \right\rVert}}
\newcommand{\heading}{\underline}
 
\DeclareMathOperator{\Kons}{Kons}
\DeclareMathOperator{\charac}{char}
 
\theoremstyle{definition}
\newtheorem{theorem}{Satz}[section]
\newtheorem{corollary}[theorem]{Folgerung}
\newtheorem{proposition}[theorem]{Proposition}
\newtheorem{lemma}[theorem]{Lemma}
\newtheorem{definition}[theorem]{Definition}
\newtheorem{example}[theorem]{Beispiel}
\newtheorem*{axiom}{Axiom}

 
\theoremstyle{remark}
\newtheorem*{remark}{Bemerkung}
 
\hfuzz=5pt 
 
\title{Skript Algebra}
\author{Lukas Metzger}
\date{\today}
 
\begin{document}
	\maketitle
	
	\setcounter{section}{-1}
	\section{Konstruktion mit Zirkel und Lineal}
	
	\begin{example}[Konstruktion des regelmäßigen 5-Ecks]
		Anleitung zur Konstruktion
		\begin{center}
			\includegraphics[width=0.8\linewidth]{bilder/bild1.png}
		\end{center}
	\end{example}

	\heading{Erste Frage:} Gegeben $n \in \setN$, kann ich das regelmäßige $n$-Eck konstruieren?
	
	\heading{Beispielproblem:} Betrachte Das $5$-Eck, sei $a$ die Kantenkänge und $s$ die Sekantenlänge.
	
	Dann ist $\frac{s}{a} \notin \setQ$.
	
	\begin{proof}
		Angenommen $\frac{s}{a}$ wäre in $\setQ$. Dann schreibe $\frac{s}{a} = \frac{p}{q}$ mit $p,q \in \setN$. Dann gibt es also eine Länge $d \in \setR$, so dass $s$ und $a$ beides ganzzahlige Vielfache von $d$ sind. $\exists n,m \in \setN$  $a = n \cdot d, s= m \cdot d$.
		
		Betrachte/Erweitere die Konstruktion des $5$-Ecks und erhalte kleines (blaues) $5$-Eck wie gezeichnet mit Sekantenlänge $s' = a$ und Kantenlänge $a' = s - a$.
		
		\begin{center}
			\includegraphics[width=0.8\linewidth]{bilder/bild2.png}
		\end{center}
	
		Dann sind aber sowohl $a'$ als auch $s'$ wieder Vielfache von $d$. Das Verfahren kann ich wiederholen und erhalte immer kleinere $5$-Ecke, deren Größe nach $0$ konvergiert, wo Kanten- und Sekantenlänge ganzzahlige Vielfache von $d$ sind. $\lightning$
	\end{proof}

	\heading{Weitere Konstruktionsprobleme:}
	\begin{itemize}
		\item $3$-Teilung des Winkels
		\item Verdoppelung des Würfels (d.h. Verdoppelung des Volumens)
		\item Quadratur des Kreises (Gegeben ein Kreis, konstruiere Quadrat mit demselben Flächeninhalt)
	\end{itemize}

	\heading{Wiederholung:} Was kann ich mit Zirkel und Lineal eigentlich machen?
	
	Antwort: $3$ Konstruktionen
	\begin{enumerate}
		\item Gegeben Punkte $a_1, a_2, b_1, b_2$ der Ebene, betrachte die Geraden $\overline{a_1 a_2}$ und $b_1 b_2$ und erhalte Schnittpunkt $\overline{a_1 a_2} \cap \overline{b_1 b_2}$.
		\item Gegeben Punkte $a_1, a_2, b_1, b_2, b_3$ der Ebene betrachte Kreis $K(b_1, \dabs{b_2 - b_3})$ um $b_1$ mit Radius $\dabs{b_2 - b_3}$ und erhalte die Schnittpunkte $\overline{a_1 a_2} \cap K(b_1, \dabs{b_2 - b_3})$
		\item Gegeben Punkte $a_1, a_2, a_3, b_1, b_2, b_3$, erhalte Schnittpunkte $K(a_1, \dabs{a_2 - a_3}) \cap K(b_1, \dabs{b_2 - b_3})$
	\end{enumerate}

	\begin{definition}
		Sei $M \subset \setR^2$ eine Menge, $p \in \setR^2$ ein Punkt.
		
		Sage: $p$ ist aus $M$ mit Zirkel und Lineal konstruierbar, falls es Kette von Mengen gibt
		\begin{equation*}
			M = M_1 \subseteq M_1 \subseteq \dots \subseteq M_n \ni p
		\end{equation*}
		Wobei $\forall i$ die Menge $M_i$ entsteht aus $M_{i-1}$ durch Hinzunahme der Punkte die durch einen Konstruktionsschritt entstehen.
	\end{definition}

	\heading{Historie:} Einen Durchbruch bei der Lösung dieser Probleme gab es erst, als man begann, die Punkte des $\setR^2$ mit komplexen Zahlen zu identifizieren.
	
	\begin{remark}
		Frage nach der Konstruierbarkeit macht nur Sinn, wenn $M$ mindestens $2$ Punkte enthält $\leadsto$ Häufig $M = \{0,1\} \subset \setC$.
	\end{remark}

	\heading{In dieser Sprache}
	\begin{itemize}
		\item Konstruktionsproblem: $n$-Eck ist äquivalent zu, kann ich die $n$-ten Einheitswurzeln $e^{\frac{i 2 \pi}{n}}$ aus $M = \{ 0,1 \}$ konstruieren?
		Ist $e^{\frac{2\pi i}{n}} \in \Kons(\{0,1\})$?
		\item Verdopplung des Würfels $\Leftrightarrow$ Ist $\sqrt[3]{2} \in \Kons(\{0,1\})$
		\item Quadratur des Kreises $\Leftrightarrow$ Ist $\sqrt{\pi} \in \Kons(\{0,1\})$
		\item $3$-teilung des Winkels $\Leftrightarrow$ Ist für gegebenes $\varphi \in (0,2\pi)$ $e^{\frac{i \varphi}{3}} \in \Kons(\{ 0, 1, e^{i \varphi} \})$
	\end{itemize}

	\heading{Zentrale Beobachtung}
	
	Sei $M \subset \setC$ eine Menge die $0$ und $1$ enthält. Sei $\Kons(M)$ die Menge der aus $M$ konstruierbaren Punkte.
	
	Dann ist $\Kons(M) \subset \setC$ ein Unterkörper.
	
	\textit{Dazu zu prüfen:} Konstruierbarkeit von Summen, Differenzen, Produkten, Quotienten \dots.
	
	\heading{Zusammenfassung/zentrales Thema der Vorlesung}
	
	Körpererweiterung / wie können Körper ineinander enthalten sein?	
	
	\section{Körpererweiterungen}
	
	\subsection{Ultrakurzwiederholung zentraler Begriffe}
	
	\begin{definition}[Gruppe]
		Eine Gruppe ist eine Menge $G$ zusammen mit einer Abbildung $m: G \times G \to G$ so dass folgendes gilt:
		\begin{enumerate}
			\item Assoziativ: $\forall a,b,c \in G \, m(m(a,b), c) = m(a, m(b,c))$
			\item Neutrales Element: $\exists n \in G \forall a \in G: m(n,a) = m(a,n) = a$
			\item Inverse Elemente: $\forall a \in G \exists b \in G: ab = ba$ und dieses Produkt ist neutrales Element wie in $2)$
		\end{enumerate}
	\end{definition}

	\begin{lemma}[Elementare Eigenschaften von Gruppen]
		Für jede Gruppe gilt:
		\begin{itemize}
			\item Das neutrale Element ist eindeutig
			\item Inverse Elemente sind eindeutig
		\end{itemize}
	\end{lemma}

	\begin{definition}[Abelsche Gruppe]
		Nenne Gruppe $(G,m)$ Abelsch, falls $\forall a,b \in G: m(a,b) = m(b,a)$.
	\end{definition}

	\heading{Notation:} Statt $m$ schreibt man oft $+$ oder $\cdot$, wobei $+$ hauptsächlich für Abelsche Gruppen verwendet wird.
	
	\begin{example}
		Beispiele für Gruppen:
		\begin{itemize}
			\item Abelsche Gruppen: $(\setZ, +)$, $(\setZ / p\setZ, +)$, $(Vektorraum, +)$
			\item Nicht-Abelsche Gruppen: Sei $M$ eine Menge mit $>2$ Elementen. Die bijektiven Abbildungen $M \to M$ mit der Hintereinanderausführung ist eine nicht-Abelsche Gruppe.
			
			Sei $K$ ein Schiefkörper, z.B. $K = \setR, \setC, \setH$. Sei $K^* K \setminus \{0\}$. Dann ist $(K^*, \cdot)$ eine Gruppe.
			\item Nicht-Beispiel: $G = \setR^3$. Ich erhalte durch das Kreuzprodukt keine Gruppenkonstruktion.
		\end{itemize}
	\end{example}

	\begin{definition}[Ring]
		Ein Ring ist eine Menge $R$ mit $2$ Verknüpfungen $+$ und $\cdot$ so dass gilt:
		\begin{itemize}
			\item $(R,+)$ ist eine Abelsche Gruppe
			\item Distributivgesetz: $\forall a,b,c \in T (a+b) \cdot c = ac + bc$ und $a(b+c) = ab + ac$
			\item $(R \setminus 0, \cdot)$ ist fast Gruppe nämlich assoziativ und es existiert ein neutrales Element
		\end{itemize}
	\end{definition}

	\begin{example}
		Beispiele für Ringe:
		\begin{itemize}
			\item $\setR, \setZ / n\setZ$, Polynome, $\setZ$
			\item Funktionen auf $\setR / \setC$
			\item holomorphe/stetige/$C^\infty$/reell analytische lokal quadratintegrierbare Funktionen bilden ebenfalls einen Ring
		\end{itemize}
	\end{example}

	\begin{remark}
		Mit Ringen kann ich fast rechnen wie mit Zahlen, aber ACHTUNG
		\begin{itemize}
			\item Nicht jedes Element in $R \setminus 0$ hat ein multiplikatives Inverses
			\item Ich kann aus $a \cdot b = 0$ und $a \neq 0$ im Allgemeinen nicht folgern, dass $b = 0$
			\item Ich kann aus $ab = ac$ und $a \neq 0$ im allgemeinen nicht folgern, dass $b = c$ ist
		\end{itemize}
	\end{remark}

	\begin{definition}[Nullteiler]
		Sei $R$ ein Ring, $a \in R \setminus \{0\}$. Falls $b \neq 0$ existiert mit $a \cdot b = 0$, nenne ich $a$ einen Nullteiler.
		
		Ringe ohne Nullteiler heißen Nullteilerfrei oder Integritätsringe.
	\end{definition}

	\begin{definition}[Abelscher Ring]
		Ein Ring heißt abelsch, falls $\forall a,b \in R \; ab = ba$.
	\end{definition}

	\begin{remark}
		In der Literatur heißen unsere Ringe oft Ringe mit $1$.
	\end{remark}

	\begin{example}
		Beispiele zu Nullteilern
		\begin{itemize}
			\item $\setR, \setZ$ sind nullteilerfrei
			\item $\setZ / n \setZ$ ist nullteilerfrei $\Leftrightarrow$ $n$ ist Prim
			\item Polynome sind nullteilerfrei
			\item Stetige Funktionen sind nicht nullteilerfrei
		\end{itemize}
	\end{example}

	\begin{remark}
		Sei $R$ ein Ringe. Die Menge der Elemente, die ein multiplikatives Inverses haben, wir mit $R^*$ bezeichnet.
		
		\begin{itemize}
			\item $\setZ^* = \{ 1, -1 \}$
			\item $(\setZ / n \setZ)^* = \{ [x] \mid x \text{ ist teilerfremd zu } n \}$
			\item $(C^\infty(\setR))^* = \{ f: \setR \to \setR \mid \text{$f$ ist $C^\infty$ und hat keine Nullstelle} \}$
		\end{itemize}
	\end{remark}

	\begin{remark}
		Sei $R$ ein Ring, $x$ eine Variable. Dann bezeichne mit $R[x]$ die Polynome mit Koeffizienten in $R$ und Variable $x$.
		
		\begin{itemize}
			\item $1x + 2 \in \setZ[x]$
			\item $\displaystyle\frac{\pi}{4} \cdot x^2 \notin \setZ[x]$
		\end{itemize}
	\end{remark}

	\begin{definition}[Schiefkörper]
		Schiefkörper sind Ringe $R$ wobei $R^* = R \setminus \{ 0 \}$
	\end{definition}

	\begin{definition}[Körper]
		Ein Körper ist ein Schiefkörper, der auch noch kommutativ ist.
	\end{definition}

	\begin{example}
		Beispiele für Körper und Schiefkörper
		\begin{itemize}
			\item Quaternionen sind Schiefkörper
			\item $\setQ, \setR, \setC, \setZ / p \setZ$ sind Körper
			\item $\Kons(\{ 0 , 1 \})$ ist Unterkörper von $\setC$
			\item Die Menge der Rationale Funktionen über einem Körper bilden wieder einen Körper
		\end{itemize}
	\end{example}

	\subsection{Algebraische und transzendente Elemente}
	
	Sei $L$ ein Körper und $k \subset L$ ein Unterkörper (z.B. $L = \setC, k \subset \setR$ oder $L = \setR, k = \setQ$).
	
	Im Fall $k = \setQ, L = \setR$ wissen wir, dass es in $\setR$ sehr unterschiedliche Elemente gibt.
	
	\begin{itemize}
		\item $\sqrt{7}$ \textellipsis algebraisch
		\item $\pi, e$ \textellipsis transzendent
	\end{itemize}

	\begin{definition}
		Situation wie oben. Sei $a \in L$ gegeben. Nenne $a$ algebraisch über $k$ falls es ein Polynom gibt $f \in k[x]$ und $f \neq 0$ so dass $f(a) = 0$.
	\end{definition}

	\begin{remark}
		Nicht algebraische Elemente heißen transzendent.
	\end{remark}

	\begin{example}
		Beispiele für algebraische und transzendente Zahlen
		\begin{itemize}
			\item $\sqrt{7}$ ist algebraisch über $\setQ$, denn $f(\sqrt{7}) = 0$ mit $f(x) = x^2 - 7$
			\item $\pi$ ist nicht algebraisch über $\setQ$ (Lindemann, 1844)
		\end{itemize}
	\end{example}

	\begin{remark}
		In $\setR$ gibt es praktisch keine Zahlen, die algebraisch über $\setQ$ sind.
		
		Wir wissen $\setQ$ ist abzählbar, also sind auch die Polynome mit Koeffizienten in $\setQ$ abzählbar. Jedes Polynom hat aber nur endlich viele Nullstellen. Das heißt die Menge der algebraischen Zahlen ist abzählbar, also eine Nullmenge im Sinne der Integrationstheorie.
	\end{remark}

	\begin{example}
		Körpererweiterung $\setR \subset \setC$ - Beobachte: $i$ ist algebraisch über $\setR$, denn $f(i) = 0$ wobei $f(x) = x^2 + 1$
		
		$z = i + 1$ ist Algebraisch mit $f(x) = (x - 1)^2 + 1$
		
		$z = a + bi$ ist Algebraisch mit $f(x) = \left( \frac{(x - a)}{b} \right)^2 + 1$
		
		$\Rightarrow$ Jede komplexe Zahl ist algebraisch über $\setR$
	\end{example}

	\begin{definition}
		Eine Körpererweiterung $k \subset L$ heißt algebraisch, falls jedes $a \in L$ algebraisch über $k$ ist.
		
		Ansonsten nenne Körpererweiterung transzendent.
	\end{definition}

	\begin{remark}
		Sei $k \subset L$ eine Körpererweiterung, sei $a \in L$ algebraisch über $k$ und sei $f \in k[x]$ ein Polynom $\neq 0$ mit $f(a) = 0$.
		
		Solche Polynome gibt es viele, wir interessieren uns für $f$'s mit mimimalem Grad. Wenn so ein $f$ gegeben ist:
		\begin{equation*}
			f = a_n x^n + a_{n-1} x^{n-1} + \dots + a_0
		\end{equation*}
		dann dividiere durch $a_n$ und erhalte Polynom
		\begin{equation*}
			\hat{f} = x^n + \frac{a_{n-1}}{a_n} x^{n-1} + \dots + \frac{a_0}{a_n} \in k[x]
		\end{equation*}
		mit $a$ als Nullstelle.
		
		Falls $\hat{f}$ und $\overline{f}$ in $k[x]$ zwei normierte Polynome von minimalem Grad sind mit $\hat{f}(a) = \overline{f}(a) = 0$, dann betrachte Polynom $(\hat{f} - \overline{f}) \in k[x]$. Dann gilt
		\begin{equation*}
			(\hat{f} - \overline{f})(a) = \hat{f}(a) - \overline{f}(a) = 0 - 0 = 0
		\end{equation*}
		und der Grad von $(\hat{f} - \overline{f})$ ist kleiner als der Grad von $\hat{f}$. Weil aber der Grad von $\hat{f}$ minimal war, folgt: $\hat{f} = \overline{f}$.
	\end{remark}

	\begin{theorem}
		Sei $k \subset L$ eine Körpererweiterung, sei $a \in L$ algebraisch über $k$. Dann gibt es genau ein Polynom $f \in k[x] \setminus \{ 0 \}$ so dass gilt:
		\begin{enumerate}
			\item $f(a) = 0$
			\item $\deg f$ ist minimal unter den Graden der Polynome die $a$ als Nullstelle haben:
			\begin{equation*}
				\deg (f) = \min \{ \deg g \mid g \in k[x] \setminus \{0 \}, g(a) = 0 \}
			\end{equation*}
			\item $f$ ist normiert (d.h. Leitkoeffizient $= 1$)
		\end{enumerate}
		Nenne dieses $f$ das Minimalpolynom von $a$ über $k$.
		
		Die Zahl $\deg f$ wird als Grad von $a$ über $k$ bezeichnet, in Symbolen $[a: k]$
	\end{theorem}

	\begin{remark}
		Sei $k \subset L$ Erweiterung, $a \in L$ algebraisch über $k$. Falls $[a:k] = 1$, dann $a \in k$.
	\end{remark}

	\heading{Mehr Beispiele für Körpererweiterungen}
	
	Sei $k \subset L$ eine Körpererweiterung, sei $(L_i)_{i \in I}$ eine Menge von Zwischenkörpern, d.h. $k \subseteq L_i \subseteq L$.
	
	Dann ist auch $K \coloneqq \bigcap_{i \in I} L_i$ ein Körper.
	
	\heading{Nutzanwendung:} Sei $A \subset L$ irgendeine Teilmenge. Sei $(L_i)_{i \in I}$ die Menge der Zwischenkörper $k \subseteq L_i \subseteq L$ so dass $\forall i: A \subset L_i$. Dann betrachte $K$ und es gilt:
	\begin{itemize}
		\item $k \subseteq K \subset L$, also $K$ ist Zwischenkörper
		\item $A \subseteq K$
		\item $K$ ist der kleinste Zwischenkörper der $A$ enthält
	\end{itemize}

	\begin{remark}
		Bezeichne $K$ mit $k(A)$ und sage $k(A)$ entsteht aus $k$ durch Adjunktion der Elemente von A.
	\end{remark}

	\heading{Spezialfall:} $A = \{ a \}$ dann schreibe ich $k(a)$. Das ist dann der kleinste Unterkörper von $L$, der sowohl $k$ als auch $a$ enthält.
	
	\begin{definition}[Einfache Körpererweiterung]
		Eine Körpererweiterung $k \subset L$ heißt einfach, falls $a$ existiert, so dass $L = k(a)$.
	\end{definition}

	\begin{definition}[Grad der Körpererweiterung]
		\begin{equation*}
			[L:k] = \dim_k L \qquad \text{Grad der Körpererweiterung}
		\end{equation*}
	\end{definition}

	\heading{Beispiele}
	\begin{equation*}
		[\setC : \setR] = 2 \qquad [\setR: \setQ]  = \infty
	\end{equation*}
	
	\begin{theorem}
		Sei $L/k$ eine Körpererweiterung, $a \in L$ dann gilt
		\begin{equation*}
			[a: k] = [k(a): k]
 		\end{equation*}
	\end{theorem}

	\begin{proof}
		Falls $a$ tanszendent, dann sind $1,a,a^2,\dots$ $k$-linear unabhängig, also ist $\dim_k k(a) = \infty$.
		
		Betrachte also den Fall, wo $a$ algebraisch ist mit Minimalpolynom $f(x) = x^n + b_{n-1} + \dots + b_0 \in k[x]$.
		
		\heading{Klar ist:} Die Elemente $1,a,a^2, \dots, a^{n-1} \in k(a)$ sind linear unabhängig, denn jede lineare Relation gäbe ein Polynom $g(x)$ vom Grad $< n$ mit $g(a) = 0$ $\lightning$.
		
		\heading{Also:} $\dim_k k(a) \geq n$
		
		Um Gleichheit zu zeigen, genügt es zu zeigen, dass $\langle 1, a, a^2, \dots, a^{n-1} \rangle_k \eqqcolon \tilde{k}$ bereits $k(a)$. Klar ist $\tilde{k} \in k(a)$. Wegen der Minimalität von $k(a)$ genügt es für die Umkehrrichtung zu zeigen, dass $\tilde{k}$ ein Körper ist.
		
		Klar ist $0,1 \in \tilde{k}$.
		
		Zu zeigen ist Abgeschlossenheit unter Addition/Subtraktion (hier klar wegen Vektorraum) und unter Multiplikation/Division (noch nicht klar).
		
		\heading{Zwischenbehauptung:} Sei $s = \sum_{i=0}^{n-1} \lambda_i a^i \in \tilde{k}$ ein beliebiges Element. Dann ist $a \cdot s \in \tilde{k}$.
		
		Wir wissen:
		\begin{equation*}
			a \cdot s = \underbrace{\sum_{i=0}^{n-2} \lambda_i a^{i+1}}_{\in \tilde{k}} + \lambda_{n-1} a^n
		\end{equation*}
		Ein Blick auf das Minimalpolynom zeigt:
		\begin{equation*}
			a^n = - \sum_{i = 0}^{n-1} b_i \cdot a^i \in \tilde{k}
		\end{equation*}
		
		\heading{Konsequenz:} Wenn $s,t \in \tilde{k}$ beliebig sind, dann $s \cdot t \in \tilde{k}$, also gilt die Abgeschlossenheit unter Multiplikation.
		
		\heading{Letzte Aufgabe:} Existenz von multiplikativen Inversen. Sei also $s \in \tilde{k}, s \neq 0$ gegeben. Wegen abgeschlossenheit unter Multiplikation ist $s, s^2, s^3, \dots$ wieder in $\tilde{k}$. Also ist $1,s, \dots, s^n$ linear abhängig $\Rightarrow$ $s$ ist algebraisch über $k$.
		
		Sei $p(x) = x^m + p_{m-1} \cdot x^{m-1} + \dots + p_0$ das Minimalpolynom.
		
		\heading{Beobachtung:} $p_0 \neq 0$, denn sonst könnte ich $x$ ausklammern, $p$ wäre nicht minimal. Damnach kann ich schreiben:
		\begin{align*}
			&0 = p(s) = s^m + p_{m-1} s^{m-1} + \dots + p_0 \\
			\Leftrightarrow& - p_0 = s(s^{m-1} + p_{m-1} s^{m-1} + \dots + p_1) \\
			\Leftrightarrow& \frac{1}{s} = \underbrace{\frac{1}{-p_0}}_{\in k} \underbrace{(s^{m-1 + p_{m-1} s^{m-2} + \dots + p_1)}}_{\in \tilde{k} \text{ wegen Abg. unter Mult.}} \in \tilde{k}
		\end{align*}
	\end{proof}

	\begin{corollary}""
		\begin{enumerate}
			\item Wenn $[a:k] = n$, dann ist $k(a) = \{ \lambda_0 + \lambda_1 a + \dots + \lambda_{n-1} a^{n-1} \mid \lambda_i \in k \}$
			\item Wenn $[a:k] < \infty$, dann ist $k(a)/k$ algebraisch
		\end{enumerate}
	\end{corollary}

	\begin{example}
		Sei $L = \setC, k \subset C$ ein Unterkörper, sei $b \in k$ und $a = \sqrt{b}$. Dann gilt:
		\begin{equation*}
			[k(a): k] = \begin{cases}
				2 & \text{falls $a \notin k$} \\
				1 & \text{falls $a \in k$}
			\end{cases}
		\end{equation*}
	\end{example}

	\begin{proposition}[Umkehrung der Beobachtung]
		\label{prop:umkehrungDerBeobachtung}
		Sei $L/k$ eine Körpererweiterung von Grad $2$. Dann entsteht $L$ durch Adjunktion einer Quadratwurzel.
	\end{proposition}

	\begin{lemma}
		Sei $L/k$ eine algebraische Körpererweiterung, so dass der Erweiterungsgrad $[L:k]$ eine Primzahl ist. Dann ist die Erweiterung einfach, das heißt $\exists a \in L: L = k(a)$.
	\end{lemma}

	\begin{proof}
		Übung
	\end{proof}

	\begin{proof}\textit{(von Proposition \ref{prop:umkehrungDerBeobachtung})}
		Wähle $a \in L$ wie im Lemma. Dann ist klar $[a:k] = 2$. Also existieren $\lambda_1, \lambda_0 \in k$, so dass $a^2 + \lambda_1 a + \lambda_0 = 0$ ist. Also:
		\begin{equation*}
			a \in \underbrace{\frac{- \lambda_1}{2}}_{\in k} \pm \underbrace{\sqrt{\left( \frac{\lambda_1}{2} \right)^2 - \lambda_0}}_{=b}
		\end{equation*}
		Weil $a$ und $b$ sich nur um Elemente von $k$ unterscheiden, ist $k(a) = k(b)$. Das Element $b$ ist aber Quadratwurzel!
	\end{proof}

	\begin{remark}
		Falls $\charac(k) = 2$ ist, muss man die Lösungsformel richtig hinschreiben.
	\end{remark}

	\begin{theorem}
		Sei $k \subseteq L \subseteq M$ eine Kette von Körpern. Dann ist
		\begin{equation*}
			[M:k] = [M:L] \cdot [L:k]
		\end{equation*}
	\end{theorem}

	\begin{proof}(nur im Fall, wo $[M:L] < \infty$ und $[L:k] < \infty$)

		Wähle Basis $m_1, \dots, m_a$ für $M$ als $L$-Vektorraum und $l_1, \dots, l_b$ für $L$ als $k$-Vektorraum.
		
		\heading{Behauptung:} Dann bilden die Elemente $(m_i \cdot l_j)_{i,j}$ eine Basis von $M$ als $k$-Vektorraum.
		
		\heading{Erzeugendensystem:} Sei $m \in M$ gegeben. Dann ist $m$ schreibbar als 
		\begin{equation*}
			m = \sum_{i = 1}^{a} \lambda_i \cdot m_i
		\end{equation*}
		mit $\lambda_i \in L$.
		
		Dann kann ich jedes $\lambda_i$ schreiben als
		\begin{equation*}
			\lambda_i = \sum_{j = 1}^{b} \mu_j^i \cdot l_j
		\end{equation*}
		mit $\mu_j \in k$.
		
		Einsetzten zeigt $m$ kann geschrieben werden als $k$-Linearkombination der Produkte $m_i \cdot l_j$.
		
		\heading{Lineare Unabhängigkeit:} Sei eine lineare Relation
		\begin{equation*}
			0 = \sum_{i,j} \mu_j{ij} \cdot (m_i \cdot l_j)
		\end{equation*}
		gegeben, wobei $\mu_j{ij} \in k$. Dann gilt
		\begin{equation*}
			0 = \sum_{i} \underbrace{\left( \sum_{j} \mu_{ij} \cdot l_j \right)}_{\in L} \cdot m_i
		\end{equation*}
		Weil die $m_i$ per Wahl aber $L$-linear unabhängig sind folgt für alle $i$ $\sum_{j} \underbrace{\mu_{ij}}_{\in k} \cdot l_j = 0$.
		
		Weil die $l_j$ per Wahl aber $k$-linear unabhängig sind, ist $\forall i \forall j \mu_{ij} = 0$.
	\end{proof}

	\begin{corollary}
		Wenn eine Kette von Körpererweiterungen gegeben ist, $k \subseteq L \subseteq M$ und wenn $[M:k] < \infty$ dann ist $[L:k] < \infty$ und sogar ein Teiler von $[M:k]$.
	\end{corollary}

	\begin{theorem}
		Sei $L/k$ eine Körpererweiterung, dann ist äquivalent:
		\begin{enumerate}
			\item $[L:k] < \infty$
			\item $L$ ist algebraisch über $k$, und es gibt endlich viele $a_1, \dots, a_n \in L: L = k(a_1, \dots, a_n)$
			\item Es gibt endlich viele $a_1 \dots, a_n \in L$, die algebraisch über $k$ sind und $L = k(a_1, \dots, a_n)$
		\end{enumerate}
	\end{theorem}

	\begin{proof}
		\heading{$1 \Rightarrow 2$:} Sei $s \in L$ beliebig. Dann sind $1,s,s^2, \dots, s^{[L:k]}$ linear abhängig, also ist $s$ algebraisch über $k$. Das heißt $L/k$ ist algebraisch.
		Um $a_1, \dots, a_n$ zu finden, wähle Vektorraumbasis von $L$ über $k$.
		
		\heading{$2 \Rightarrow 3$:} trivial
		
		\heading{$3 \Rightarrow 1$:} Betrachte
		\begin{equation*}
			\underbrace{k}_{\eqqcolon k_0} \subseteq \underbrace{k(a_1)}_{\eqqcolon k_1} \subseteq \underbrace{k(a_1, a_2)}_{\eqqcolon k_2} \subseteq \dots \subseteq \underbrace{k(a_1, \dots, a_n)}_{\eqqcolon k_n}
		\end{equation*}
		Dann klar: $\forall i: a_i \text{ ist algebraisch über } k_{i-1}$ (sogar algebraisch über $k_0$) also $[k_i: k_{i-1}] < \infty$, dann $k_i = k_{i-1}(a_i)$ und $[L:k] = \prod_i [k_i: k_{i-1}] < \infty$.
	\end{proof}

	\begin{lemma}[Nutzanwendung (Transitivität der Algebraizität)]
		Sei $k \subseteq L \subseteq M$ eine Kette von Körpererweiterungen. Falls $L/k$ algebraisch ist und $M/L$ algebraisch ist, dann ist $M/k$ algebraisch.
	\end{lemma}

	\begin{proof}
		Sei $m \in M$ gegeben. Ziel: $m$ ist algebraisch über $k$.
		
		$m$ ist algebraisch über $L$, das heißt es hat ein Minimalpolynom
		\begin{equation*}
			f(x) = \sum_{i=0}^a l_i \cdot x^i \in L[x]
		\end{equation*}
		Wir wissen auch: Jedes der $l_i$ ist algebraisch über $k$.
		
		Betrachte jetzt den Zwischenkörper $L' = k(l_0, \dots, l_a)$. Dann ist $L'/k$ endlich und $m$ ist algebraisch über $L'$, also ist $m \in L'(m)$ und $L'(m)/L'$ ist endlich. Damit ist $L'(m)/k$ endlich, also algebraisch.
	\end{proof}

	\begin{proposition}
		Sei $k \subseteq L$ eine Körpererweiterung. Sei
		\begin{equation*}
			\overline{k} \coloneqq \{ a \in L \mid a \text{ ist algebraisch über } k \}
		\end{equation*}
		Dann ist $\overline{k}$ ein Körper.
		
		Man nennt $\overline{k}$ den algebraischen Abschluss von $k$ in $L$.
	\end{proposition}

	\begin{proof}
		Klar ist, dass $0,1 \in \overline{k}$ sind. Wir müssen klären, ob mit $a,b \in \overline{k}$ auch $a+b, a-b, a \cdot b$ und gegebenenfalls für $\frac{1}{a} \in \overline{k}$ sind. Das ist aber klar, denn all diese Elemente liegen in $k(a,b)$. Nach Satz ist $k(a,b)$ algebraisch über $k$.
	\end{proof}

	\begin{remark}
		Achtung: Es gibt einen anderen Begriff von (absolutem) algebraischen Abschluss, der nicht von einem Oberkörper $L \supseteq k$ abhängt.
	\end{remark}

	\subsection{Lösungsformel für Polynome}
	
	\heading{Wissen aus der Schule:} Quadratische Gleichungen in einer Variable haben Lösungsformel.
	
	\heading{Wissen seit der Renaissance:} Haaben Formeln für Gleichungen von Grad 3 und 4.
	
	\heading{Beispiel:} $x^3 + a x^2 + bx + c = 0$ Setze:
	\begin{equation*}
		h = - \frac12 c + \frac16 a b - \frac{1}{24} a^3
	\end{equation*}
	\begin{align*}
		w_1 &= \sqrt{-3 (a^2 b^2 - 4 a^3c - 4b^3 + 18abc - 27c^2)} \\
		w_2 &= \sqrt[3]{h + \frac{1}{18} w_1} \\
		w_2 &= \sqrt[3]{h - \frac{1}{18} w_1}
	\end{align*}
	
	Dann ist
	\begin{equation*}
		x = - \frac13 a + w_2 - w_3
	\end{equation*}
	eine Lösung, wenn die Wurzeln $w_2, w_3$ so gewählt sind dass $w_2 w_3 = \frac18 a^2 - \frac13 b$.
	
	\heading{Frage:} Gibt es eine Lösungsformel für Gleichungen vom Grad 5?
	
	\heading{Bescheidener:} Kann ich die Lösung überhaupt hinschreiben? (als komplizierten Ausdruck in Wurzeln/Polynomen)
	
	\begin{definition}
		Sei $L/k$ eine Körpererweiterung, nenne diese Erweiterung Radikalerweiterung, falls es $a_1, \dots, a_n$ und $m_1, \dots, m_n \in \setN$ gibt, so dass
		\begin{enumerate}
			\item $L = k(a_1, \dots, a_m)$
			\item $\forall i a_i^{m_i} \in k(a_1, \dots, a_{i-1})$ also $a_i$ ist die $m_i$-te Wurzel eines Elementes aus $k(a_1, \dots a_{i-1})$.
		\end{enumerate}
	\end{definition}

	\heading{Was bedeutet das?}
	\begin{enumerate}
		\item $a_1^{m_1} \in k$ Also $k(a_1) = \langle 1 , a_1, a_1^2, \dots, a_1^{m_1 - 1} \rangle_k$
		\item $a_2^{m_2} \in k$ Also $k(a_1, a_2) = \langle 1 , a_2, a_2^2, \dots, a_2^{m_2 - 1} \rangle_{k(a_1)}$
		\item \textellipsis
	\end{enumerate}

	\heading{Bescheidene Frage, präzise formuliert:} Gegeben ein Polynom
	\begin{equation*}
		f(x) = \sum_{i=1}^{n} a_i x^i \in \setQ[x] \text{ oder } \setR[x]
	\end{equation*}
	gibt es dann eine Radikalerweiterung $L/\setQ(a_0, \dots, a_n)$ (beziehungsweise $L/\setR$) so dass $f$ in $L$ eine Nullstelle hat? Gerne $L \subseteq \setC$.
	
	\section{Ringe}
	
	\heading{Warum Ringe betrachten?} Gegeben eine Körpererweiterung $L/k$ und $a \in L$ und ich suche das Minimalpolynom $f_a(x) \in k[x]$.
	
	Häufig findet man $g \in k[x]$ mit $g(a) = 0$ und muss dann entscheiden ob $g$ das Minimalpolynom ist. Das ist gar nicht leicht!
	
	\heading{Beobachtung:} Polynomdivision zeigt:
	\begin{equation*}
		g(x) = s(x) \cdot f_a(x) + \operatorname{rest}(x)
	\end{equation*}
	wobei $\deg \operatorname{rest}(x) < \deg f_a(x)$. $a$ einsetzen ergibt
	\begin{equation*}
		\underbrace{g(a)}_{= 0} = s(a) \cdot \underbrace{f_a(a)}_{=0} + \operatorname{rest}(a) \Rightarrow \operatorname{rest}(a) = 0
 	\end{equation*}
 	$\Rightarrow \operatorname{rest}(x) \equiv 0$
 	
 	$\Rightarrow g(x) = s(x) \cdot f_a(x)$.
 	
 	\heading{Wir sehen:} Das Minimalpolynom ist ein Teiler von $g$ im Ring der Polynome.
 	
 	\heading{Ziel:} Wir müssen Teilbarkeit verstehen!
 	
 	\subsection{Teilbarkeit}
 	
 	\begin{definition}
 		Sei $R$ ein Ring. Dann bezeichne mit $R[x]$ den Ring der Polynome mit Variable $x$ und Koeffizienten aus $R$.
 	\end{definition}
 
 	\heading{Warnung:} Polynome geben Funktionen $R \to R$ aber Polynome sind nicht Funktionen.
 	
 	\begin{definition}
 		Sei $f \in R[x]$ ein Polynom. Dann definiere den Grad von $f$ wie üblich.
 	\end{definition}
 
 	\begin{lemma}
 		Sei $R$ ein Integritätsring, $f,g \in R[x]$. Dann ist
 		\begin{equation*}
	 		\deg(f \cdot g) = \deg(f) + \deg(g)
 		\end{equation*}
 	\end{lemma}
 
 	\begin{proof}
 		Sei $n_f = \deg(f)$ und $n_g = \deg(g)$ schreibe
 		\begin{align*}
	 		f(x) &= a_f \cdot x^{n_f} + (\text{kleinere Terme}), a_f \neq 0 \\
	 		g(x) &= a_g \cdot x^{n_g} + (\text{kleinere Terme})
 		\end{align*}
 		
 		Dann ist
 		\begin{equation*}
 			(f \cdot g)(x) = a_f \cdot a_g \cdot x^{n_f + n_g} + (\text{kleinere Terme})
 		\end{equation*}
 		und weil $R$ ein Integritätsring ist, ist $a_f \cdot a_g \neq 0$, also $\deg(f \cdot g) = n_f + n_g$.
 	\end{proof}
 
 	\begin{corollary}
 		Sei $R$ ein Integritätsring. Dann ist $R[x]$ selbst wieder ein Integritätsring.
 	\end{corollary}
 
 	\begin{proof}
 		Seien $f,g \in R[x] \setminus \{0\}$.
 		
 		Wir müssen zeigen: $f \cdot g \not\equiv 0 \in R[x]$ $(*)$.
 		
 		Falls $\deg f = \deg g = 0$, folgt $(*)$ weil $R$ ein Integritätsring ist.
 		
 		Ansonsten folgt $(*)$, weil $\deg f \cdot g = \deg f + \deg g > 0$.
 	\end{proof}
 
 	\heading{Ausblick:} Dann ist $(R[x])[y]$ auch wieder ein Integritätsring. Und natürlich ist $(R[x])[y] \simeq R[x,y]$.
 	
 	\begin{corollary}
 		Sei $R$ ein Integritätsring. Dann ist $(R[x])^* = R^*$.
 	\end{corollary}
 
 	\begin{proof}
 		Sei $f(x) \in (R[x])^*$, das heißt $\exists g(x) \in R[x]: f \cdot g \equiv 1$.
 		
 		$\Rightarrow \deg f + \deg g = \deg 1 = 0$
 		
 		$\Rightarrow \deg f = 0$, also ist Polynom $f$ konstant, ebenso für $g$. 
 	\end{proof}
 
 	\begin{remark}
 		Per Induktion folgt auch $(R[x_1, \dots, x_n])^* = R^*$
 	\end{remark}
 
 	\begin{definition}
 		Sei $R$ ein Ring, seien $s, r \in R$ Elemente. Ich sage: $s$ ist Teiler von $r$ (in Symbolen $s \mid r$), wenn es $a \in R$ gibt, so dass $s \cdot a = r$.
 	\end{definition}
 
 	\begin{lemma}
 		Sei $R$ ein Integritätsring, seien $s,r$ Elemente. Dann ist äquivalent
 		\begin{enumerate}
 			\item $\exists \varepsilon \in R^*, s = \varepsilon \cdot r$
 			\item $s \mid r$ und $r \mid s$
 		\end{enumerate}
 	
 	Wenn diese Bedingungen erfüllt sind, nenne ich $s$ und $r$ assoziiert (in Symbolen $s \sim r$).
 	\end{lemma}
 
 	\begin{proof}
 		$1) \Rightarrow 2) \checkmark$
 		
 		$2) \Rightarrow 1)$ Aus $s \mid r$ und $r \mid s$ $\Rightarrow a,b \in R: s \cdot a = r$ und $r \cdot b = s$.
 		
 		$\Rightarrow (r \cdot b) \cdot a \Rightarrow r(ba - 1) = 0$
 		
 		Da $R$ Integritätsring ist: $\Rightarrow ba = 1 \qquad \Rightarrow b,a, \in R^*$
 	\end{proof}
 
 	\begin{definition}
 		Sei $R$ ein Integritätsring, seien $s,r \in R$ Elemente. Dannn nenne $s$ einen echten Teiler von $r$ (in Symbolen $s \parallel r$) falls gilt:
 		\begin{enumerate}
 			\item $s \mid r$
 			\item $s \notin R^*$
 			\item $r$ und $s$ sind nicht assoziiert
 		\end{enumerate}
 	\end{definition}
 
 	\begin{definition}
 		Sei $R$ ein Integritätsring. Ein Element $r \in R$ heißt irreduzibel, falls $r \notin R^*$ und falls $r$ keine echten Teiler hat.
 	\end{definition}
 
 	\begin{example}
 		Die irreduziblen Elemente von $R = \setZ$ sind exakt $\pm(\text{Primzahl})$.
 	\end{example}
 
 	\begin{lemma}
 		Sei $R$ ein Integritätsring. Seien $r, s, t, s_1, s_2, u, v \in R$. Dann gilt:
 		\begin{enumerate}
 			\item $r \mid r$
 			\item $r \mid s$ und $s \mid t \Rightarrow r \mid t$
 			\item $r \mid s_1$ und $r \mid s_2 \Rightarrow r \mid (s_1 + s_2)$
 			\item $r \mid s_1$ und $r \mid (s_1 + s_2) \Rightarrow r \mid s_2$
 			\item $r \mid s$ und $u \mid v \Rightarrow ru \mid sv$
 		\end{enumerate}
 	\end{lemma}
 
 	\heading{Nächstes Ziel:} In $\setZ$ ist jede Zahl darstellbar als Produkt von Primzahlen und die Darstellung ist eindeutig bis auf Reihenfolge und Vorzeichen.
 	
 	\heading{Wunschtraum:} Sei $R$ ein Integritätsring. Dann ist jedes Element eindeutig darstellbar als Produkt von irreduziblen Elementen.
 	
 	\begin{example}
 		Betrachte $R = \setZ[\sqrt{-5}] = \{ a + b \cdot \sqrt{-5} \mid a,b \in \setZ \} \subset \setC$
 		
 		Dieser Ring ist ein Unterring von $\setC$ und deshalb Nullteilerfrei und
 		\begin{equation*}
 			9 = 3 \cdot 3 = \underbrace{(2 + \sqrt{-5})(2 - \sqrt{-5})}_{2^2 - (\sqrt{-5})^2}
 		\end{equation*}
 		Die Elemente $3, 2 \pm \sqrt{-5}$ sind irreduzibel und nicht zueinander assoziiert.
 	\end{example}
 
 	\begin{definition}
 		Sei $R$ ein Integritätsring. Eine Teilerkette ist eine Folge $(r_i)_{i \in \setN}$ von Elementen aus $R$, so dass $\forall i \; r_{i+1} \mid r_i$. Ich sage, im Ring $R$ gilt der Teilerkettensatz für Elemente, falls in jeder Teilerkette die stärkere Bedigung $r_{i+1} \parallel r_i$ nur endlich oft gilt.
 	\end{definition}
 
 	\begin{example}
 		Im Ring $\setZ$ gilt der Teilerkettensatz für Elemente, denn falls $r_{i+1} \parallel r_i$ ist, dann gilt $\abs{r_{i+1}} < \abs{r_i}$.
 		
 		Analog im Polynomring mit $\deg$ statt $\abs{\cdot}$.
 	\end{example}
 
 	\begin{theorem}
 		Sei $R$ ein Integritätsring in dem der Teilerkettensatz für Elemente gilt. Dann ist jedes $r \in R, r \notin R^*, r \neq 0$ als Produkt von endliche vielen irreduziblen Elementen darstellbar.
 	\end{theorem}
 
 	\begin{proof}(Noether Rekursion)
 		Wir wollen zeigen, dass $M = \{ r \in R \mid r \notin R^*, r \neq 0 \text{ und $r$ nicht als Produkt von endlich vielen irreduziblen darstellbar} \}$ leer ist. Widerspruchsbeweis: angenommen $M \neq \emptyset$.
 		
 		Beobachtungen:
 		\begin{enumerate}
 			\item $\forall r \in M$ $r$ ist nicht irreduziblen (denn sonst wäre $r$ eine Darstellung), also hat $r$ echte Teiler
 			\item $\exists r \in M$, so dass alle echten Teiler von $r$ nicht mehr in $M$ liegen (denn sonst nehme echten Teiler aus $M$, widerhole das Verfahren, erhalte unendliche Teilerkette wo ich in jedem Schritt echte Teiler habe $\lightning$ zur Annahme)
 		\end{enumerate}
 	
 		Also gegeben $r$ wie in Beobachtung 2), dann ist jeder echte Teiler als Produkt von endlich vielen irreduziblen darstellbar, also auch $r$ selbst. (Schreibe $r = r_1 \cdot r_2$ mit $r_1, r_2$ echte Teiler. Dann $r_1 = a_1 \cdots a_n, r_2 = b_1 \dots b_m$ mit $\forall i,j a_i, b_j$ irreduzibel dann $r = a_1 \dots a_n b_1 \dots b_m$) $\lightning$.
 	\end{proof}
 
 	\begin{definition}
 		Sei $R$ ein Integritätsring, sei $r \in R, r \notin R^*, r \neq 0$. Seien
 		\begin{equation*}
	 		r = a_1 \cdots a_n = b_1 \cdots b_m
 		\end{equation*}
 		zwei Darstellungen von $r$ als Produkt von endlich vielen Irreduziblen.
 		
 		Nenne die Darstellung äquivalent, falls gilt
 		\begin{enumerate}
 			\item gleich lang: $n = m$
 			\item $\exists$ Permutation $\sigma \in S_n$ und Einheiten $\varepsilon_1 \cdots \varepsilon_n \in R^*$ so dass $\forall i: a_i = \varepsilon_i \cdot b_{\sigma(i)}$
 		\end{enumerate}
 	\end{definition}
 
 	\begin{remark}
 		In Ringen, in denen der Teilerkettensatz gilt, sind Darstellungen nicht immer äquivalent! Zum Beispiel $R = \setZ{\sqrt{-5}}$.
 		
 		Das Problem ist, dass die irreduziblen Elemente in $\setZ[\sqrt{-5}]$ nicht unbedingt prim sind.
 	\end{remark}

	\begin{definition}
		Sei $R$ ein Integritätsring, $r \in R, r \neq 0$ ein Element. Nenne $r$ prim falls $\forall a,b \in R$
		\begin{equation*}
			r \mid (a \cdot b) \qquad \Longrightarrow \qquad r \mid a \text{ oder } r \mid b
		\end{equation*}
	\end{definition}

	\begin{example}
		In $R = \setZ[\sqrt{-5}]$ ist $(2 + \sqrt{-5})$ irreduzibel, aber  nicht prim, denn $(2 + \sqrt{-5}) \mid 3 \cdot 3$ aber $(2 + \sqrt{-5}) \nmid 3$.
	\end{example}

	\begin{lemma}[Elementare Rechenregeln für Prim-Elemente]
		Sei $R$ ein Integritätsring, $p,q \in R$
		\begin{enumerate}
			\item $p$ prim $\Rightarrow$ $p$ irreduzibel
			\item $p$ prim, $p \sim s$ $\Rightarrow$ $s$ prim
			\item $p,q$ prim und $p \mid q \Rightarrow p \sim q$
			\item $p$ prim und $p \mid a_1 \cdots a_n \Rightarrow \exists i \; p \mid a_i$
		\end{enumerate}
	\end{lemma}

	\begin{proof}
		zu 1)
		
		Sei $p$ prim. Angenommen $p$ habe echten Teiler $a \in R$. Dann sei $b \in R$ so dass $p = a \cdot b$, insbesondere $p \mid ab$. Also $p \mid a$ oder $p \mid b$. oBdA gelte $p \mid a$.
		
		Also $\exists h \in R, p \cdot h = a$. Einsetzen liefert
		\begin{equation*}
			p = p \cdot h \cdot b \qquad \Longleftrightarrow \qquad p(1-hb) = 0 \qquad \underset{\text{$R$ Integritätsring}}{\Longleftrightarrow} \qquad 1 = h \cdot b
		\end{equation*}
		$\Rightarrow$ $b$ ist eine Einheit, kein echter Teiler.
	\end{proof}

	\begin{theorem}
		Im Ring $\setZ$ ist jedes irreduzible Element auch prim.
	\end{theorem}

	\begin{proof}
		Angenommen es existiert in $\setZ$ ein irreduzibles Element $p$, das nicht prim ist. Dann ist $-p$ irreduzibel und auch nicht prim. Wir können also oBdA annehmen $p > 0$. Wir können auch annehmen das $p$ das kleinste positive, irreduzible Element ist, das nicht prim ist.
		
		Also $\exists a,b \in \setN: p \mid a \cdot b$ aber $p \nmid a$ und $p \nmid b$.
		
		Division mit Rest liefert
		\begin{align*}
		 	a &= x \cdot a + a' & \text{wobei $a' < p$}\\
		 	b &= y \cdot p + b' & \text{wobei $b' < p$}
		\end{align*}
		Sehe sofort $p \nmid a'$ und $p \nmid b'$.
		
		Sehe auch $a \cdot b = x y p^2 + (x b' + a' y) p + a'b'$ also $p \mid a' b'$.
		
		Wähle also $a,b$ so, dass $ab$ minimal ist, und dann ist $a < p, b < p, ab < p^2$.
		
		Finde $h \in \setN: p \cdot h = a \cdot b$.
		
		Sei jetzt $p'$ ein irreduzibler Teiler von $h, p' > 0$. Dann existiert $h' > 0, h = p' \cdot h'$ und $p' \leq h < p$. Nach Wahl von $p$ (kleinstes irreduzibles das nicht prim ist) ist $p'$ prim und $p \cdot p' \cdot h' = a \cdot b$.
		
		Also gilt $p' \mid a \cdot b \underset{\text{$p' prim$}}{\Rightarrow} p' \mid a$ oder $p' \mid b$. oBdA gelte $p' \mid a$. Finde also $a' < a$ so dass $p' \cdot a' = a$. Einsetzen liefert 
		\begin{equation*}
			p \cdot p' \cdot h' = p' \cdot a' \cdot b \underset{\text{$\setZ$ Integritätsring}}{\Longrightarrow} p \cdot h' = a' b \qquad\Longrightarrow\qquad p \mid a' b
		\end{equation*}
		
		Da $a' b < ab$ ist gilt nach Wahl von $a \cdot b$ ($a,b$ Gegenbeispiel zur Prim-Eigenschaft mit minimalem Produkt) also $p \mid a'$ oder $p \mid b$. Da $a' \mid a$ ist folgt $p \mid a$ oder $p \mid b$. $\lightning$
	\end{proof}

	\begin{theorem}
		Sei $R$ ein Integritätsring. Dann ist äquivalent:
		\begin{enumerate}
			\item Jedes $r \in R, r \notin R^*, r \neq 0$ ist als Produkt von endlich vielen Irreduziblen darstellbar und je zwei Darstellungen sind äquivalent.
			\item In $R$ gilt der Teilerkettensatz für Elemente und alle irreduziblen sind prim.
		\end{enumerate}
		Falls diese Eigenschaften gelten, nenne $R$ faktoriell oder UFD.
	\end{theorem}

	\begin{proof}
		\heading{1) $\Rightarrow$ 2)}
		
		\textit{Teilerkettensatz}: Sei $(r_i)_{i \in \setN}$ eine Teilerkette. Sei $i$ so dass $r_{i+1} \parallel r_i$ das heißt $\exists h: h \notin R^*, h \neq 0: r_{i+1} \cdot h = r_i$.
		
		Nach Annahme, kann $r_i, r_{i+1},h$ als Produkt von endlich vielen irreduziblen geschrieben werden
		\begin{align*}
			r_i &= a_1 \cdot a_n \\
			r_{i+1} &= b_1 \cdots b_m \\
			h &= c_1 \cdots c_k
		\end{align*}
		Dann gilt
		\begin{equation*}
			\underbrace{b_1 \cdots b_m}_\text{Darstellung von $r_{i+1}$} \cdot c_1 \cdots c_k = \underbrace{a_1 \cdots a_n}_\text{Darstellung von $r_i$}
		\end{equation*}
		Da alle Darstellungen äquivalent sind, folgt $n = m + k > m$.
		
		Also in der Teilerkette gibt es höchstens endlich viele echte Teiler, nämlich höchstens so viele, wie eine (jede) Darstellung von $r_1$ lang ist. $\Rightarrow$ Teilerkettensatz gilt
		
		\textit{Irreduzibel $\Rightarrow$ Prim}: Sei $r$ irreduzibel und seien $a,b \in R \setminus \{ 0 \}$ so dass $r \mid ab$. Also existiert $h \in R \setminus \{ 0 \}$, so dass $r \cdot h = a \cdot b$. Wir wissen $h,a,b$ haben Darstellung
		\begin{equation*}
			a = a_1 \cdots a_n, \qquad b = b_1 \cdots b_m, \qquad h = h_1 \cdots h_k
		\end{equation*}
		Also
		\begin{equation*}
			r \cdot h_1 \cdots h_k = a_1 \cdots a_n \cdot b_1 \cdots b_m
		\end{equation*}
		zwei Darstellungen von $a \cdot b$. Per Annahme sind diese Darstellungen äquivalent also $\exists i: r \sim a_i$ oder $\exists j: r \sim b_j$
		
		$\Rightarrow r \mid a$ oder $r \mid b$. Also ist $r$ prim.
		
		\heading{2) $\Rightarrow 1)$}
		
		Wir haben schon bewiesen: Teilerkettensatz $\Rightarrow$ Darstellbarkeit, es fehlt noch die Äquivalenz $\forall r \in R, r \notin R^*, r \neq 0$ und für alle Darstellungen $r =  a_1 \cdots a_n \underset{(*)}{=} b_1 \cdots b_m$ mit $n \neq m$ gilt, dass beide Darstellungen äquivalent sind.
		
		\textit{Beweis per Induktion über $n$}
		
		Induktionsanfang: $n = 1: a_1 = b_1 \cdots b_m$
		
		Per Annahme ist $a_1$ prim, also $\exists j: a_1 \mid b_j$.
		
		Rechenregeln: $a_1 \sim b_j$, insbesondere sind alle $b_k, k \neq j$ schon Einheiten. $\Rightarrow m = 1 = j$ (da die Faktoren in der Darstellung irreduzibel und keine Einheiten sind).
		
		Induktionsschritt: Sei die Aussage für alle Zahlen $< n$ schon bewiesen.
		
		Wieder gilt $a_1 \mid b_1 \cdots b_m \Rightarrow \exists j: a_1 \sim b_j$. oBdA sei $j = 1$ also existiert eine Einheit $\varepsilon \in R^*$ so dass $a_1 = \varepsilon b_1$.
		
		$R$ ist also Integritätsring, kann also in $(*)$ kürzen, erhalte
		\begin{equation*}
			a_2 \cdots a_n = (\varepsilon b_2) \cdot b_3 \cdots b_m
		\end{equation*}
		Per Induktionsannahme sind diese Darstellungen äquivalent.
	\end{proof}

	\begin{corollary}
		$\setZ$ ist faktoriell.
	\end{corollary}

	\begin{corollary}
		Alle Körper sind faktoriell.
	\end{corollary}

	\begin{theorem}[Gauß]
		Wenn $R$ ein faktorieller Ring ist, dann auch $R[x]$.
		
		Und damit auch $(R[x])[y] = R[x,y]$ und auch $R[x_1, \dots, x_n] \; \forall n \in \setN$.
	\end{theorem}

	\begin{proof}
		Wir müssen zeigen:
		\begin{enumerate}
			\item In $R[x]$ gilt der Teilerkettensatz
			\item Je zwei Darstellungen sind äquivalent
		\end{enumerate}
	
		\heading{zu 1):} Wenn $r(x),s(x) \in R[x]$ und $r(x) \parallel s(x)$, dann $\deg r(x) < \deg s(x)$ oder $\exists a \in R \setminus R^*, a \neq 0: a \cdot r(x) = s(x)$.
		
		$\Rightarrow$ alle Koeffizienten von $s$ werden von $a$ geteilt. In $R$ gilt aber der Teilerkettensatz!
		
		\textit{Hausaufgabe:} Also gilt der Teilerkettensatz auch in $R[x]$.
		
		\heading{zu 2):} Widerspruchsbeweis! Angenommen es gibt $r(x) \in R[x], r \neq 0, r \notin R[x]^* = R^*$ so dass $r$ zwei Darstellungen hat, die nicht äquivalent sind
		\begin{equation*}
			r(x) = p_1(x) \cdots p_\alpha(x) = q_1(x) \cdots q_\beta(x) \qquad\qquad\qquad (*)
		\end{equation*}
		Ich kann oBdA einige Annahmen treffen
		\begin{itemize}
			\item $\deg r(x)$ ist minimal unter allen Polynomen die nicht äquivalente Darstellungen haben
			\item die irreduziblen Polynome $p_1, \dots, p_\alpha, q_1, \dots, q_\beta$ sind nach Graden sortiert also $\deg p_1 \geq \deg p_2 \geq \dots \geq \deg p_\alpha$ und $\deg q_1 \geq \deg q_2 \geq \dots \geq \deg q_\beta$
			\item $\deg q_1 \geq \deg p_1$
		\end{itemize}
	
		Sei $n \coloneqq \deg p_1, m = \deg q_1$. Seien $a,b$ die Leitkoeffizienten von $p_1$ beziehungsweise $q_1$. Das heißt:
		\begin{align*}
			p_1 &= a \cdot x^n + (\text{lot}) \\
			p_1 &= b \cdot x^m + (\text{lot})
		\end{align*}
		
		Beobachtungen:
		\begin{itemize}
			\item $\deg r(x) > 0$, denn sonst wären $r(x)$ und alle $q_i(x), p_j(x)$ konstant, also in $R$. Per Annahme das $R$ faktoriell ist müssten die Darstellungen dann äquivalent sein.
			
			$\Rightarrow$ $n > 0$ und $m > 0$
			
			\item Angenommen es gäbe $j$: $p_1 \sim q_j$. Dann könnte ich in $(*)$ auf beiden Seiten $p_1$ kürzen und erhielte Polynom von Grade $(\deg r(x)) - n < \deg r(x)$, das zwei nicht äquivalente Darstellungen hat $\lightning$ zur Minimalität von $\deg r(x)$.
		\end{itemize}
	
		Betrachte Hilfspolynom:
		\begin{equation*}
			s(x) = \underbrace{\left[ b \cdot p_1(x) \cdot x^{m-n} - a \cdot q_1(x) \right]}_{\deg < \deg q_1(x)} \cdot q_2 \cdots q_\beta \tag{\sun}
		\end{equation*}
		
		Wir erhalten zwei offensichtliche Fälle
		\begin{enumerate}
			\item $s(x) = 0$: Dann ist
			\begin{equation*}
				b \cdot p_1(x) \cdot x^{m-n} - a \cdot q_1(x)
			\end{equation*}
			\item $s(x) \neq 0$: Wir sehen $\deg s(x) < \deg r(x)$. Also sind je zwei Darstellungen von $s(x)$ äquivalent! Schreibe $s(x)$ um:
			\begin{align*}
				s(x) &= b \cdot p_1(X) x^{m-n} \cdot q_2 \cdots q_\beta - a \underbrace{q_1 \cdots q\beta}_{r(x)} \\
				&= b \cdot p_1 x ^{m-n} \cdot q_2 \cdot q_\beta - a \cdot p_1 \cdots p_\alpha \\
				&= p_1(x) \left[ b \cdot x^{m-n} \cdot q_2(x) \cdots q_\beta(x) - a \cdot p_2(x) \cdots p_\alpha(x) \right] \tag{\leftmoon}
			\end{align*}
		\end{enumerate}
	
		Wir können die Ausdrücke $(\sun)$ und $(\leftmoon)$ verfeinern zu Produkten von irreduziblen indem wir die Ausdrücke in $[\dots]$ als Produkt von irreduziblen schreiben. Diese Darstellungen von $s(x)$ müssen dann äquivalent sein.
		
		\textit{Konsequenz:} In der Darstellung von $(\sun)$ muss es einen Faktor geben, der zu $p_1$ assoziiert ist. Da $p_1 \nsim 1_2 \dots p_1 \nsim q_\beta$ muss $p_1$ ein Primfaktor vom $[\dots]$-Ausdruck in $(\sun)$ sein.
		\begin{equation*}
			\Rightarrow \qquad p_1 \mid (b p_1 \cdot x^{m-n} - a q_1) \qquad \Rightarrow p_1 \mid a q_1
		\end{equation*}
		
		Insgesamt ergibt sich in jedem der beiden Fälle:
		\begin{equation*}
			\exists h \in R[x]: \quad p_1(x) \cdot h(x) = a \cdot q_1(x) \tag{\bell}
		\end{equation*}
		
		\textit{Beobachte:} Wenn $a \in R^*$, dann $p_1 \mid q_1$ und $p_1 \sim q_1$ $\lightning$. Also ist $a \in R \setminus R^*, a \neq 0$.
		
		\textit{Zwischenbehauptung (Beweis später):} Sei $p \in R$ irreduzibel. Dann ist das konstante Polynom $p \in R[x]$ prim.
		
		\textit{Anwendung der Zwischenbehauptung:} Schreibe $a$ als Produkt von Irreduziblen. Wenn jetzt $p$ einer der irreduziblen Faktoren ist, dann $p \mid p_1 \cdot h$.
		
		$\Rightarrow p \mid p_1$ oder $p \mid h$. $p \mid p_1$ kann nicht sein, denn $p_1$ ist irreduzibel, hat also überhaupt keine echten Teiler.
		
		Also kann ich aus $(\bell)$ $p$ herausteilen und erhalte
		\begin{equation*}
			p_1 \cdot \frac{h}{p} = \frac{a}{p} q_1
		\end{equation*}
		Das geht mit jedem Primfaktor von $a$ erhalte also am Ende:
		\begin{equation*}
			p_1 \cdot \frac{h}{a} = q_1 \qquad\Rightarrow p_1 \mid q_1 \qquad\Rightarrow p_1 \sim q_1 \qquad\Rightarrow \lightning
		\end{equation*}
	
		\textit{Zwischenbehauptung (jetzt der Beweis):} Sei $p \in R$ irreduzibel. Dann ist das konstante Polynom $p \in R[x]$ prim.
		
		Sei $p \in R$ irreduzibel. Ich zeige die Kontraposition: wenn $a(x), b(x) \in R[x]$ Polynome sind mit $p \nmid a(x)$ und $p \nmid b(x)$ $\Rightarrow p \nmid (a \cdot b)(x)$
		
		Seien also $a(x), b(x)$ gegeben. Schreibe
		\begin{align*}
			a(x) &= a_0 + a_1 x + \dots + a_n x^n \\
			b(x) &= b_0 + b_1 x + \dots + b_m x^m
		\end{align*}
		
		\textit{Erinnere:} $p \mid a(x) \Leftrightarrow \forall i: p \mid a_i$
		
		Kann also minimale Indizes $i$ und $j$ wählen, so dass $p \nmid a_i$ und $p \nmid b_j$. Betrachte Produktpolynom $(a \cdot b)(x)$ und rechne den Koeffizienten von $x^{i+j}$ im Produktpolynom aus. Dieser Koeffizient ist
		\begin{equation*}
			\gamma \coloneqq \sum_{\substack{\alpha + \beta = i + j \\ \alpha, \beta \in \setN}} a_\alpha \cdot b_\beta
		\end{equation*}
		
		In dieser Summe sind alle Summanden durch $p$ teilbar, weil stets $\alpha < i$ oder $\beta < j$ mit der Ausnahme des Summanden $\alpha = i, \beta = j, (= a_i \cdot b_j)$.
		
		Weil $R$ faktoriell ist per Annahme und $p \in R$ deshalb prim ist $\Rightarrow p \nmid a_i \cdot b_j$
		
		$\Rightarrow p \nmid \gamma \qquad \Rightarrow p \nmid (a \cdot b)(x)$
	\end{proof}

	\heading{Was tun wir mit faktoriellen Ringen?}
	
	Sei $R$ ein faktorieller Ring, betrachte die Äquivalenzrelation $a \sim b \Leftrightarrow a \text{ assoziiert zu } b$
	
	Wähle Repräsentantensystem $P \subset R$ für die irreduziblen Elemente (= zu jedem irreduziblen $a \in R$ gibt es genau ein $b \in P$ mit $a \sim b$)
	
	Wenn dann irgendein $a \in R$ gegeben ist, dann kann ich schreiben
	\begin{equation*}
		a = \varepsilon \cdot \prod_{p \in P} p^{\alpha_p}
	\end{equation*}
	wobei $\varepsilon \in R^*, \alpha_p \in \setN$ und alle bis auf endlich viele $\alpha_p = 0$.
	
	Teilbarkeit wird dann ganz einfach. Seien $a, b \in R$
	\begin{equation*}
		a = \varepsilon_a \cdot \prod_{p \in P} p^{\alpha_{a,p}}, \qquad b = \varepsilon_b \cdot \prod_{p \in P} p^{\alpha_{b,p}}
	\end{equation*}
	und
	\begin{align*}
		a \mid b &\Leftrightarrow \forall p \in P: \alpha_{a,p} \leq \alpha_{b,p} \\
		a \parallel b &\Leftrightarrow (\forall p \in P: \alpha_{a,p} \leq \alpha_{b,p}) \quad\&\quad (\exists p \in P: \alpha_{a,p} < \alpha_{b,p})\\
		a \sim b &\Leftrightarrow \forall p \in P: \alpha_{a,p} = \alpha_{b,p}
	\end{align*}
	
	\heading{Weiter mit Grundschulstoff:}
	
	Sei $R$ ein Integritätsring, seien $a,b \in R \setminus R^*, a \cdot b \neq 0$
	
	\begin{enumerate}
		\item Ein Element $c \in R$ heißt größter Gemeinsamer Teiler wenn gilt $c \mid a$ und $c \mid b$ und wenn für jedes andere $c'$ mit $c' \mid a$ und $c' \mid b$ gilt $c' \mid c$.
		\item Ein Element $c \in R$ heißt kleinstes gemeinsames Vielfaches, wenn $a \mid c$ und $b \mid c$ ist und für alle $c' \in R$ mit $a \mid c'$ und $b \mid c'$ gilt $c \mid c'$.
	\end{enumerate}

	\begin{theorem}
		Sei $R$ faktoriell. Seien $a,b \in R$ dann existieren ggT und kgV.
	\end{theorem}

	\begin{proof}
		Wähle Repräsentantensystem $P \subset R$. Schreibe
		\begin{equation*}
			a = \varepsilon_a \cdot \prod_{p \in P} p^{\alpha_{a,p}}, \qquad b = \varepsilon_b \cdot \prod_{p \in P} p^{\alpha_{b,p}}
		\end{equation*}
		
		Setze
		\begin{equation*}
			\operatorname{ggT}(a,b) \coloneqq \prod_{p \in P} p^{\min(\alpha_{a,p}, \alpha_{b,p})}
		\end{equation*}
		und
		\begin{equation*}
			\operatorname{kgV}(a,b) \coloneqq \prod_{p \in P} p^{\max(\alpha_{a,p}, \alpha_{b,p})}
		\end{equation*}
		
		Blick nach oben zeigt, dass dies exakt die Bedingungen erfüllt.
	\end{proof}

	\begin{theorem}
		Seien $f,g \in k[x]$ Polynome. Betrachte Divisionsreste
		\begin{align*}
			f &= q_1 \cdot g + r_1 \tag{1}\\
			g &= q_2 \cdot r_1 + r_2 \tag{2}
		\end{align*}
		
		Definiere dann induktiv Polynome $r_n$ als Divisionsrest
		\begin{equation*}
			r_{n-2} = q_n \cdot r_{n-1} + r_n \tag{n}
		\end{equation*}
		
		\textit{Beobachtung:} Die Grade der Polynome $r_1, r_2, \dots$ werden immer kleiner. Der Prozess stoppt also nach endlich vielen Schritten das heißt irgendwann geht die Division auf. Es existiert also $n \in \setN$ so dass
		\begin{equation*}
			r_{n-1} = q_{n+1} \cdot r_n + 0 \tag{n+1}
		\end{equation*}
		
		Dann ist $r_n = \operatorname{ggT}(f,g)$.
	\end{theorem}

	\begin{proof}
		\begin{enumerate}
			\item Wenn $t$ ein gemeinsamer Teiler von $f,g$ ist
			\begin{align*}
				\overset{(1)}{\Longrightarrow} t \mid r_1 && \dots && \overset{(n)}{\Longrightarrow} t \mid r_n \\
				\overset{(2)}{\Longrightarrow} t \mid r_2
			\end{align*}
			\item Andere Richtung analog:
			\begin{align*}
				(n+1) &\Longrightarrow r_n \mid r_{n-1} \\
				(n) &\Longrightarrow r_n \mid r_{n-2} \\
				\vdots \\
				(2) &\Longrightarrow r_n \mid g \\
				(1) &\Longrightarrow r_n \mid f
			\end{align*}
			Da $k[t]$ faktoriell ist genügen 1) + 2) um $r_n = \operatorname{ggT}$ zu zeigen.
		\end{enumerate}
	\end{proof}

	\subsection{Der Quotientenkörper eines Integritätsrings}
	
	\heading{Ziel:} Gegeben ein Ring $R$, suche einen möglichst kleinen Körper $k$ s.d. $R \subset k$ (besser: so dass es einen injektiven Ringmorphismus $R \hookrightarrow k$ gibt). Wir denken an $\setZ \hookrightarrow \setQ$.
	
	\heading{Beobachtung:} So etwas kann es nicht geben, wenn $R$ Nullteiler hat! Betrachte also nur Integritätsringe.
	
	\begin{definition}
		Sei $R$ ein Integritätsring. Ein Quotientenkörper von $R$ ist ein Körper $k$ zusammen mit einem injektiven Ringmorphismus $\varphi: R \to k$ so dass folgende (universelle) Eigenschaft gilt: Wann immer $\Phi: R \to L$ ein injektiver Ringmorphismus in einen Körper ist, dann gibt es genau einen Körpermorphismus $\eta: k \to L$ so dass das folgende Diagramm kommutiert.
		
		\begin{center}
			\begin{tikzpicture}
				\node(R1) at (0,0){$R$};
				\node[right = 2 of R1](k){$k$};
				\node[below = 1 of R1](R2){$R$};
				\node[below = 1 of k](L){$L$};
				
				\draw[->] (R1) -- (k) node[midway,above]{$\varphi$};
				\draw (R1) -- (R2) node[midway,left]{$\mathbbm{1}_R$};
				\draw[->] (R2) -- (L) node[midway,below]{$\Phi$};
				\draw[->] (k) -- (L) node[midway,right]{$\exists ! \eta$};
			\end{tikzpicture}
		\end{center}
	\end{definition}

	\begin{remark}
		Körpermorphismen $k \overset{\eta}{\to} L$ sind immer injektiv! Denn wäre $a \in k \setminus \{0\}, a \in \ker(\eta)$. Dann
		\begin{equation*}
			1_L = \eta(1_k) = \eta(a \cdot a^{-1}) = \underbrace{\eta(a)}_{= O_L} \cdot ?
		\end{equation*} 
		Widerspruch!
	\end{remark}

	\begin{theorem}
		Sei $R$ ein Integritätsring. Dann existiert ein Quotientenkörper $(k, \varphi: R \to k)$. Dieser ist eindeutig bis auf kanonische Isomorphie. Das bedeutet: Wenn $(k', \varphi': R \to k')$ ein weiterer Quotientenkörper ist, dann existiert genau ein Körperisomorphismus $\eta: k \to k'$ so das das folgende Diagramm kommutiert.
		
		\begin{center}
			\begin{tikzpicture}
				\node(R1) at (0,0){$R$};
				\node[right = 2 of R1](k1){$k$};
				\node[below = 1 of R1](R2){$R$};
				\node[below = 1 of k1](k2){$k$};
				
				\draw[->] (R1) -- (k1) node[midway,above]{$\varphi$};
				\draw (R1) -- (R2) node[midway,left]{$\mathbbm{1}_R$};
				\draw[->] (R2) -- (k2) node[midway,below]{$\varphi'$};
				\draw[->] (k1) -- (k2) node[midway,right]{$\exists ! \eta$};
			\end{tikzpicture}
		\end{center}
	\end{theorem}

	\begin{proof}		
		\heading{Eindeutigkeit:} Seien Quotientenkörper $(k, \varphi: R \to k)$ sowie $(k', \varphi': R \to k')$ gegeben. Nach der universellen Eigenschaft existiert dann genau ein Körpermorphismus $\eta: k \to k'$ so dass das folgende Diagramm kommutiert:
		\begin{center}
			\begin{tikzpicture}
				\node(R1) at (0,0){$R$};
				\node[right = 2 of R1](k1){$k$};
				\node[below = 1 of R1](R2){$R$};
				\node[below = 1 of k1](k2){$k$};
				
				\draw[->] (R1) -- (k1) node[midway,above]{$\varphi$};
				\draw (R1) -- (R2) node[midway,left]{$\mathbbm{1}_R$};
				\draw[->] (R2) -- (k2) node[midway,below]{$\varphi'$};
				\draw[->] (k1) -- (k2) node[midway,right]{$\eta$};
			\end{tikzpicture}
		\end{center}
		Wir wissen auch: Weil $k'$ Quotientenkörper ist, existiert genau ein Körpermorphismus $\eta': k' \to k$ so dass das folgende Diagramm kommutiert:
		\begin{center}
			\begin{tikzpicture}
				\node(R1) at (0,0){$R$};
				\node[right = 2 of R1](k1){$k$};
				\node[below = 1 of R1](R2){$R$};
				\node[below = 1 of k1](k2){$k'$};
				\node[below = 1 of R2](R3){$R$};
				\node[below = 1 of k2](k3){$k$};
				
				\draw[->] (R1) -- (k1) node[midway,above]{$\varphi$};
				\draw (R1) -- (R2) node[midway,left]{$\mathbbm{1}_R$};
				\draw[->] (R2) -- (k2) node[midway,below]{$\varphi'$};
				\draw[->] (k1) -- (k2) node[midway,right]{$\eta$};
				\draw[->] (R3) -- (k3) node[midway,below]{$\varphi$};
				\draw[->] (k2) -- (k3) node[midway,right]{$\eta'$};
				\draw (R2) -- (R3) node[midway,left]{$\mathbbm{1}_R$};
			\end{tikzpicture}
		\end{center}
	
		Die universelle Eigenschaft angewandt auf
		\begin{center}
			\begin{tikzpicture}
				\node(R1) at (0,0){$R$};
				\node[right = 2 of R1](k1){$k$};
				\node[below = 1 of R1](R2){$R$};
				\node[below = 1 of R2](R3){$R$};
				\node[right = 2 of R3](k3){$k$};
				
				\draw[->] (R1) -- (k1) node[midway,above]{$\varphi$};
				\draw (R1) -- (R2) node[midway,left]{$\mathbbm{1}_R$};
				\draw[->] (R3) -- (k3) node[midway,below]{$\varphi$};
				\draw[->] (k1) -- (k3) node[midway,right]{$\eta' \circ \eta$} node[midway,left]{$\mathbbm{1}_k$};
				\draw (R2) -- (R3) node[midway,left]{$\mathbbm{1}_R$};
			\end{tikzpicture}
		\end{center}
		zeigt: $\eta' \circ \eta = \mathbbm{1}_k$.
		
		Genauso folgt $\eta \circ \eta'= \mathbbm{1}_{k'}$. Also ist der Körpermorphismus $\eta'$ die Umkehrung von $\eta$.
		
		\heading{Existenz:} Ich konstruiere den Quotientenkörper wie folgt:
		\begin{enumerate}
			\item Betrachte die Menge
			\begin{equation*}
				B = \{ (a,b) \in R \times R \mid b \neq 0 \}
			\end{equation*}
			und sage $(a,b)$ ist äquivalent zu $(a', b')$ wenn gilt $a b' = a' b$. Das ist eine Äquivalenzrelation. Symmetrie und Reflexivität sind klar per Definition. Wir müssen also noch die Transitivität zeigen: Seien also Tupel gegeben so dass
			\begin{align*}
				&&(a,b) &\sim (a', b') & (a', b') &\sim (a'',b'') \\
				&\Leftrightarrow& ab' &= a'b & a'b'' &= a''b'
			\end{align*}
			Und damit dann
			\begin{equation*}
				\Rightarrow\quad ab' \cdot a'b'' = a' b \cdot a'' b'
			\end{equation*}
			Im Integritätsring falls $a' \neq 0$
			\begin{equation*}
				\Rightarrow\quad a b'' = a'' b \quad\Leftrightarrow\quad (a,b) \sim (a'',b'')
			\end{equation*}
			Falls $a' = 0$ ist der Beweis sowieso einfach.
			
			Definiere als Menge
			\begin{equation*}
				k \coloneqq B / \sim
			\end{equation*}
			
			\textit{Notation:} Die Äquivalenzklasse von $(a,b)$ wird mit $\frac{a}{b}$ bezeichnet.
			
			Betrachte die Abbildung
			\begin{equation*}
				\varphi: R \to k, a \mapsto \frac{a}{1}
			\end{equation*}
			
			Diese Abbildung ist injektiv, denn 
			\begin{equation*}
				\varphi(a) = \varphi(a') \quad\Leftrightarrow\quad \frac{a}{1} = \frac{a'}{1} \overset{\text{Def.}}{\quad\Leftrightarrow\quad} a \cdot 1 = a' \cdot 1 \quad\Leftrightarrow\quad a = a'
			\end{equation*}
			\item Definiere auf $k$ die Struktur eines Körpers mit Verknüpfungen
			\begin{align*}
				\cdot&: k \times k \to k, \quad \left(\frac{a}{b}, \frac{c}{d}\right) \mapsto \frac{ac}{bd} \\
				+&: k \times k \to k, \quad \left(\frac{a}{b}, \frac{c}{d}\right) \mapsto \frac{ad + cb}{bd} \\
			\end{align*}
			Muss noch nachrechnen: Wohldefiniertheit
			
			Das bedeutet: Gegeben $\frac{a}{b}$ und $\frac{c}{d}$ sowie $\frac{a'}{b'}$ und $\frac{c'}{d'}$ mit $\frac{a}{b} = \frac{a'}{b'}$ sowie $\frac{c}{d} = \frac{c'}{d'}$, dann gilt $\frac{ad + cb}{bd} = \frac{a'd' + c'b'}{b'd'}$
			\begin{align*}
				&\Leftrightarrow& (ad + cb) \cdot b'd' &= (a'd' + c'b') \cdot bd \\
				&\Leftrightarrow& adb'd' + cbb'd' &= a'd'bd + c'b'bd \\
				\intertext{Wir wissen $ab' = a'b$ und $cd' = c'd$} \\
				&\Leftrightarrow& 0 &= 0
			\end{align*}
			Die Addition ist wohldefiniert.
			
			\textit{Hausaufgabe:} Dasselbe für Multiplikation
			
			\textit{Lästige Rechnerei:} Diese Verknüpfungen definieren eine Körperstruktur auf $k$ so dass die Abbildung $\varphi: R \to k$ ein Ringmorphismus ist. Es gilt
			\begin{equation*}
				0_k = \frac01 \qquad 1_k = \frac11 \qquad \text{falls $a \neq 0$ dann} \left( \frac{a}{b} \right)^{-1} = \frac{b}{a}
			\end{equation*}
			
			\item Beweis der universellen Eigenschaft
			
			Sei Körper $L$ gegeben und ein injektiver Ringmorphismus $\Phi: R \to L$, dann müssen wir zeigen $\exists! \eta: k \to L$ so dass \textellipsis
			
			\heading{Eindeutigkeit:} Angenommen wir hätten $\eta$ so dass das folgende Diagramm kommutiert
			\begin{center}
				\begin{tikzpicture}
					\node(R1) at (0,0){$R$};
					\node[right = 2 of R1](k){$k$};
					\node[below = 1 of R1](R2){$R$};
					\node[below = 1 of k](L){$L$};
					
					\draw[->] (R1) -- (k) node[midway,above]{$\varphi$};
					\draw (R1) -- (R2) node[midway,left]{$\mathbbm{1}_R$};
					\draw[->] (R2) -- (L) node[midway,below]{$\Phi$};
					\draw[->] (k) -- (L) node[midway,right]{$\exists ! \eta$};
				\end{tikzpicture}
			\end{center}
			dann gilt für alle $a \in R$
			\begin{equation*}
				\eta(\varphi(a)) = \Phi(\mathbbm{1}_R(a)) \Leftrightarrow \eta\left(\frac{a}{1}\right) = \Phi(a)
			\end{equation*}
			Falls $a \neq 0$ ist gilt
			\begin{equation*}
				\eta\left( \frac{1}{a} \right) = \eta\left( \left( \frac{a}{1} \right)^{-1} \right) \overset{\text{Körpermorphismus}}{=} \eta \left( \frac{a}{1} \right)^{-1}
			\end{equation*}
			also gilt für alle $\frac{a}{b} \in k$
			\begin{equation*}
				\eta \left( \frac{a}{b} \right) = \eta \left( \frac{a}{1} \cdot \frac{1}{b} \right) = \eta\left( \frac{a}{1} \right) \cdot \eta \left(
				\frac{1}{b}\right) = \Phi(a) \cdot \left( \Phi(b) \right)^{-1}
			\end{equation*}
			also ist $\eta$ eindeutig.
			
			\heading{Existenz:} Definiere
			\begin{equation*}
				\eta: k \to L, \quad \frac{a}{b} \mapsto \Phi(a) \cdot \Phi(b)^{-1}
			\end{equation*}
			Wieder ist Wohldefiniertheit zu prüfen: Seien $\frac{a}{b} = \frac{a'}{b'}$. Wir müssen zeigen:
			\begin{align*}
				&& \Phi(a) \Phi(b)^{-1} &= \Phi(a') \Phi(b')^{-1} \\
				&\Leftrightarrow& \Phi(a) \cdot \Phi(b') &= \Phi(a') \Phi(b) \\
				&\Leftrightarrow& \Phi(ab') &= \Phi(a'b) \\
				&\Leftrightarrow& \text{Wahr,}& \text{ wegen Annahme}
			\end{align*}
			Nachrechnen: das ist ein Körperisomorphismus.
		\end{enumerate}
	\end{proof}

	\begin{example}""
		\begin{itemize}
			\item $R = \setZ$ dann ist $Q(\setZ) = \setQ$
			\item $R$ ein Körper, dann ist $Q(R) = R$
			\item $R = \setZ[2 + \sqrt{-5}]$, dann ist $Q(R) = \setQ(2 + \sqrt{-5}) \subset \setC$
			
			Grund: Wir haben eine Inklusion $R \subset \setQ(2 + \sqrt{-5})$ deshalb gibt es Körpermorphismus $Q(R) \to \setQ(2 + \sqrt{-5})$.
			
			Dieser ist surjektiv, denn $Q(R)$ enthält das Element $a = 2 + \sqrt{-5}$. Wir wissen aber $\setQ(2 + \sqrt{-5})$ ist der kleinste Körper der dieses Element enthält.
			\item Sei $R$ faktoriell. Wähle Repräsentantensystem $P \subset R$. Dann kann ich alle Elemente von $Q(R)$ auf eindeutige Weise schreiben als
			\begin{equation*}
				\varepsilon \cdot \prod_{p \in P} p^{\alpha_p}
			\end{equation*}
			wobei $\varepsilon \in R^*, \alpha_p \in \setZ$ und fast alle $\alpha_p = 0$.
		\end{itemize}
	\end{example}

	\heading{Warum das alles?}
	
	Wenn $R$ faktoriell ist, kann ich manchmal entscheiden, ob Polynome in $R[x]$ irreduzibel sind.
	
	\textit{Beispiel:} $f(x) = x^3 - 2 \in \setZ[x]$
	
	Behauptung: $f$ ist irreduzibel in $\setZ[x]$
	
	Anbenommen es gäbe einen echten Teiler, dann gäbe es einen linearen Teiler das heißt
	\begin{equation*}
		\exists a,b \in \setZ, a \neq 0: f(x) = (ax + b) g(x)
	\end{equation*}
	wobei $g(x)$ quadratisch in $\setZ[x]$.
	
	Sehe sofort: $a \in \{ \pm 1 \}, b \in \{ \pm 1, \pm 2 \}$
	
	Nachrechnen: keine dieser Möglichkeiten ist ein Teiler
	
	Der folgende Satz zeigt, dass $f$ auch in $\setQ[x]$ irreduzibel ist.
	
	\begin{theorem}[Satz von Gauß]
		Sei $R$ ein faktorieller Ring. Falls $f(x) \in R[x]$ irreduzibel als Element von $R[x]$, dann ist $f$ auch irreduzibel als Element von $Q(R)[x]$.
	\end{theorem}

	\heading{Vorbemerkung:} Sei $f \in Q(R)[x]$ irgendein Polynom. Dann existiert $a \in Q(R)$ so dass $a \cdot f(x) \subset R[x]$ und $\operatorname{ggT}(\text{Koeffizienten von } a \cdot f(x)) = 1$ (Koeffizienten sind Teilerfremd).
	
	Beweis dazu: Auf Hauptnenner bringen und durch größten gemeinsamen Teiler der Koeffizienten teilen.
	
	\begin{proof}
		Angenommen wir haben $f(x) \in R[x]$ welches als Polynom in $Q(R)[x]$ reduzibel ist. Das heißt es existieren Polynome $q(x), p(x) \in Q(R)[x]$ mit $q,p$ nicht konstant, so dass $f(x) = q(x) \cdot p(x)$.
		
		\textit{Ziel:} Schreibe $f$ als Produkt $f = q'(x) \cdot p'(x)$ wobei $q',p' \in R[x]$ echte Teiler sind.
		
		\textit{Beobachtung:} Wenn $\gamma \in R$ jeden Koeffizienten von $f$ teilt und $\gamma \notin R^*, \gamma \neq 0$ dann ist $\gamma$ ein echter Teiler von $f$ und wir sind fertig. Wir nehmen also ab sofort an, dass die Koeffizienten von $f$ teilerfremd sind.
		
		Wende Vorbemerkung auf Polynome $p(x), q(x)$ an, erhalte $a,b \in Q(R)$ so dass $a \cdot p(x) \in R[x]$ und $b \cdot q(x) \in R[x]$ und Koeffizienten dieser Polynome jeweils Teilerfremd in $R$.
		
		Durch Multiplikation erhalte Gleichung
		\begin{equation*}
			a \cdot b \cdot f(x) = a \cdot p(x) \cdot b \cdot q(x) \in R[x] \tag{$*$}
		\end{equation*}
		Beachte die linke Seite ist in $R[x]$, weil beide Faktoren der rechten Seite in $R[x]$ sind.
		
		\textit{Behauptung:} Es ist $a \cdot b \in R$.
		
		\textit{Beweis:} Angenommen $a \cdot b \notin R$ das heißt es existiert Primelement $p \in R$, welches in der Darstellung von $a \cdot b$ mit negativem Exponenten auftritt. Da aber $a \cdot b \cdot f(x) \in R[x]$ muss die Darstellung jedes Koeffizienten das Element $p$ mit positivem Exponenten enthalten. Also $p \mid \text{Koeffizienten}$ $\lightning$ zu $\operatorname{ggT(\text{Koeffizienten})} = 1$
		
		\textit{Behauptung:} Es gilt sogar $a \cdot b \in R^*$
		
		\textit{Beweis:} Angenommen $a \cdot b \notin R^*$. Dann hätte ich einen echten irreduziblen Teiler $\gamma \in R$ irreduzibel mit $\gamma \mid a \cdot b$.
		\begin{equation*}
			\Rightarrow \gamma \mid a \cdot b \cdot f(x) \qquad \Rightarrow \gamma \mid [a \cdot p(x)][b \cdot q(x)]
		\end{equation*}
		
		\textit{Erinnerung:} $\gamma \in R$ irreduzibel $\Rightarrow$ $\gamma$ prim in $R[x]$.
		
		Also gilt
		\begin{equation*}
			\gamma \mid a \cdot p(x) \qquad\text{oder}\qquad \gamma \mid b \cdot q(x)
		\end{equation*}
		oBdA sei $\gamma \mid a \cdot p(x)$ $\lightning$ zur Wahl von $a$.
		
		Damit kann ich $(*)$ umschreiben zu
		\begin{equation*}
			f(x) = \underbrace{\left[ (a \cdot b)^{-1} \cdot a \cdot p(x) \right]}_{\in R[x]} \cdot \underbrace{\left[ b \cdot q(x) \right]}_{\in R[x]}
		\end{equation*}
	\end{proof}

	\heading{Zusammenfassung:} Wir sind jetzt inder Lage, für ganzzahlige Polynome zu entscheiden, ob sie in $\setQ[x]$ irreduzibel sind. (z.B. $x^3 - 2$ ist irreduzibel in $\setQ[x]$, Folgerung $[\setQ(\sqrt[3]{2}): \setQ] = 3$ denn wir wissen jetzt, dass $x^3 - 2$ das Minimalpolynom von $\sqrt[3]{2}$ ist)
	
	\heading{Erinnerung:} Das geht so:
	
	Lagrangesche Interpolationsformel (= Polynom von Grad $\leq n$ ist durch seine Werte an $n+1$ Stellen festgelegt) Sei $k$ Körper, $f(x) \in k[x]$ Polynome von Grad $\leq n$, seien $a_1, \dots, a_{n+1} \in k$ unterschiedliche Körperlementente. Dann ist $f$ durch die Werte $f(a_i)$ eindeutig festgelegt, nämlich
	\begin{equation*}
		f(x) = \sum_{j=1}^{n+1} f(a_j) \prod_{k \neq j} \frac{x - a_k}{a_j - a_k} \eqqcolon h(x) \in k[x]
	\end{equation*}
	Dann gilt für alle $i$
	\begin{equation*}
		h(a_i) = \sum_{j = 1}^{n+1} f(a_j) \prod_{k \neq j} \frac{a_i - a_k}{a_j - a_k} = f(a_i) \prod_{k \neq i} \frac{a_i - a_k}{a_i - a_k} = f(a_i)
	\end{equation*}
	
	$\Rightarrow$ $h-f$ ist Polynom von Grad $\leq n$ mit Nullstellen $a_1, \dots, a_{n+1}$
	
	$\Rightarrow h - f = 0$
	
	Damit haben wir folgendes Verfahren, um irreduzibilität in $\setZ[x]$ und also auch in $\setQ[x]$ zu testen.
	
	Gegeben $f(x) \in \setZ[x]$ von Grad $\leq n$ so dass $\operatorname{ggT}(\text{Koeffizienten}) = 1$.
	
	Wähle $a_1, \dots, a_{n+1} \in \setZ$ so dass $f(a_i) \neq 0$ und betrachte $f(a_1), \dots, f(a_n) \in \setZ$.
	
	Wir wissen, wenn $g(x)$ ein Teiler von $f(x)$ in $\setZ[x]$ ist, dann gilt für alle $i$ $q(a_i) \mid f(a_i)$
	
	Für $g(a_i)$ gibt es also nur endlich viele Möglichkeiten.
	
	Nur endlich viele Polynome kommen als Teiler in Frage. Wir müssen also durch Polynomdivision testen, ob die Kandidatenpolynome tatsächlich Teiler sind.
	
	\begin{theorem}[Eisenstein-Kriterium]
		Sei $R$ ein faktorieller Ring, sei\begin{equation*}
			f(x) = a_0 + a_1 x + \dots + a_n x^n \in R[x]
		\end{equation*}
		mit $n > 0$ und $\operatorname{ggT}(a_0, \dots, a_n) = 1$. Falls es ein irreduzibles gibt $p \in R$ so dass $p \mid a_0, p \mid a_1, \dots, p \mid a_{n-1}$ und $p^2 \nmid a_0$. Dann ist $f$ irreduzibel in $R[x]$ und also auch in $Q(R)[x]$.
	\end{theorem}

	\begin{proof}
		Sei $f$ wie im Satz gegeben. Angenommen ich kann $f$ schreiben als Produkt
		\begin{equation*}
		f(x) = \alpha(x) \cdot \beta(x)
		\end{equation*}
		wobei $\alpha, \beta \in R[x], \deg \alpha > 0, \deg \beta > 0$.
		
		Schreibe
		\begin{align*}
			\alpha(x) &= \alpha_0 + \alpha_1 x + \dots \\
			\beta(x) &= \beta_0 + \beta_1 x + \dots
		\end{align*}
		
		Beobachte: $a_0 = \alpha_0 \cdot \beta_0$
		
		Per Annahme gilt: $p \mid a_0 \underset{\text{$R$ faktoriell}}{\Rightarrow} p \mid \alpha_0$ oder $p \mid \beta_0$.
		
		Per Annahme $p^2 \nmid a_0$ kann $p$ nicht beide Elemente teilen. Wir nehmen also $p \mid \alpha_0$ und $p \nmid \beta_0$ an.
		
		Weil $\operatorname{ggT}(a_0, \dots, a_n) = 1$ wissen wir $p$ teilt nicht alle $\alpha_i$. Sei also $i$ minimal so dass $p \nmid \alpha_i$. Wir wissen schon mal $i < n$, insbesondere $p \mid a_i$.
		
		Es ist aber
		\begin{equation*}
			a_i = \underbrace{\alpha_0 \beta_i}_\text{Vielfaches von $p$} + \underbrace{\alpha_i \beta_{i-1}}_\text{Vielfaches von $p$} + \underbrace{\alpha_2 \beta_{i-2}}_\text{Vielfaches von $p$} + \dots + \underbrace{\alpha_i \beta_0}_\text{kein Vielfaches von $p$}
		\end{equation*}
		
		$\lightning$ zu Teilbarkeitsregeln.
	\end{proof}

	\begin{remark}
		Polynome welche die Annahmen des Satzes erfüllen heißen Eisensteinpolynome.
		
		Ein Beispiel dafür ist $R = \setZ, f(x) = x^3 - 2$.
	\end{remark}

	\subsection{Hilfe bei der Anwendung des Eisenstein-Kriteriums}
	
	Sei $R$ faktoriell und $\varphi: R[x] \to S$ ein Ringmorphismus in einen Integritätsring $S$. Angenommen $\varphi$ hat die Eigenschaft dass $\forall f \in R[x]: \deg f > 0 \Rightarrow \varphi(f) \notin S^*$.
	
	Wenn jetzt ein $f \in R[x]$ gegeben ist mit $f(x) = a_0 + a_1 x + \dots + a_n x^n$ mit $n > 0$ und $\operatorname{ggT}(a_0, \dots, a_n) = 1$ und $\varphi(f)$ irreduzibel ist, dann ist $f$ irreduzibel.
	
	\begin{proof}
		Angenommen $f(x)$ sei reduzibel in $R[x]$ $\Rightarrow$ $\exists \alpha(x), \beta(x) \in R[x]$ mit $f(x) = \alpha(x) \cdot \beta(x)$ und $\deg \alpha > 0, \deg \beta > 0$. Dann gilt
		\begin{equation*}
			\varphi(f) = \varphi(\alpha \cdot \beta)  = \underbrace{\varphi(\alpha)}_{\notin S^*} \cdot \underbrace{\varphi(\beta)}_{\notin S^*}
		\end{equation*}
		Also hat $\varphi(f)$ echte Teiler in $S$ und ist damit nicht irreduzibel.
	\end{proof}

	\heading{Wie finde ich $\varphi$?}: Keine Ahnung wir müssen Rumprobieren
	
	\heading{Beispielhafte Konstruktionen}
	
	\begin{enumerate}
		\item Gegeben ein Ringmorphismus $\phi: R \to S$ (z.B. $\setZ \to \setZ / p \setZ$ oder $\setZ / p \setZ \to \setZ / p \setZ, a \mapsto a^p$)
		
		Betrachte dann Morphismus von Polynomringen
		\begin{equation*}
			\varphi: R[x] \to S[x], \sum a_i x^i \mapsto \sum \phi(a_i) x^i
		\end{equation*}
		
		\item Situation wie in 1), zusätzlich sei $s \in S$ gegeben. Betrachte
		\begin{equation*}
			\varphi^*: R[x] \to S, \sum a_i x^i \mapsto \sum \phi(a_i) s^i	
		\end{equation*}
		
		\item Situation wie in 2). Betrachte Morphismus
		\begin{equation*}
			\varphi^{\leftmoon}: R[x] \to s[x], \sum a_i x^i \mapsto \sum \varphi(a_i) (x - s)^i
		\end{equation*}
	\end{enumerate}

	\heading{Beispielhafte Nutzanwendung:} Betrachte $p \in \setN$ prim und
	\begin{equation*}
		f(x) = x^{p-1} + x^{p-2} + \dots + x + 1 \in \setZ[x]
	\end{equation*}
	Das ist kein Eisenstein Polynom.
	
	Beobachte aber auch $(x-1)f(x) = x^p -1$.
	
	Das legt nahe folgenden Morphismus zu probieren
	\begin{equation*}
		\varphi: \setZ[x] \to \setZ[x], g(x) \mapsto g(x + 1)
	\end{equation*}
	
	was ist $\varphi(f)$?
	
	\begin{equation*}
		\varphi(x^p - 1) = \varphi((x-1)f) = \underbrace{\varphi(x-1)}_{=x} \cdot \varphi(f)
	\end{equation*}
	und außerdem
	\begin{equation*}
		\varphi(x^p - 1) = (x+1)^p - 1 = \sum_{i=1}^{p} \binom{p}{i} x^i - 1
	\end{equation*}
	
	\begin{equation*}
		\Rightarrow \varphi(f) = \sum_{i = 1}^p \binom{p}{i} \cdot x^{i-1}
	\end{equation*}
	das ist ein Eisenstein Polynom.
	
	Also ist $f(x)$ irreduzibel in $\setZ[x]$, also auch in $\setQ[x]$.
	
	\heading{Ernte einfahren:} Wir können mit unseren Methoden einige Fragen beantworten.
	
	\heading{Erinnerung:} Gegeben $M \subset \setC$, eine Menge die $0,1$ enthält. $\operatorname{Konst}(M) =$ Menge der aus $M$ konstruierbaren Punkte.
	
	\begin{enumerate}
		\item $\operatorname{Kons}(M)$ ist ein Unterkörper von $\setC$
		\item Wenn $z \in \operatorname{Kons}(M) \subset \setC$, dann gibt es $n \in \setN$ so dass $[k(z): k] = 2^n$ wobei $k = \setQ(M \cup \overline{M})$ und $M = \{ \overline{m} \mid m \in M \}$.
	\end{enumerate}
	
	\begin{example}
		$z = \sqrt[3]{2}$. Ist nicht aus $M = \{ 0, 1 \}$ konstruierbar, denn in diesem Fall wäre $\overline{M} = M$ und $k = \setQ(0,1) = \setQ$ aber $[\setQ(\sqrt[3]{2}: \setQ)] = 3$, denn wir wissen: $x^3 - 2$ ist das Minimalpolynom.
	\end{example}

	Dieselbe Argumentation liefert mehr!
	
	\begin{theorem}
		Sei $\varphi \in (0,2\pi)$ so dass $e^{i \varphi} \in \setC$ transzendent ist. Dann ist der Winkel
		
		TODO winkel durch $e^{i\varphi}$
		
		nicht durch Zirkel und Lineal $3$-teilbar.
	\end{theorem}

	\begin{remark}
		Die Abbildung
		\begin{equation*}
			(0, 2\pi) \to \setC, \varphi \mapsto e^{i \varphi}
		\end{equation*}
		ist injektiv hat also überabzählbar viele Bildpunkte, es gibt aber nur abzählbar viele algebraische Zahlen. Also ist $e^{i \varphi}$ transzendent für fast alle $\varphi$.
	\end{remark}

	Für dieses Problem betrachte bei gegebenem $\varphi$ die Menge $M = \{ 0, 1, e^{i \varphi} \}$. Ist $e^{i \frac{\varphi}{3}} \in \operatorname{Kons(M)}$? Also betrachte ich
	\begin{equation*}
		k = \setQ(M \cup \overline{M}) = \setQ(z) = \setQ(e^{i \varphi})
	\end{equation*}
	Muss diskutieren: $[k(e^{i \frac{\varphi}{3}}): k]$ das ist eine $2$-er Potenz falls $e^{i \frac{\varphi}{3}}$ konstruierbar ist.
	
	Wir sehen $e^{i \frac{\varphi}{3}}$ ist Nullstelle des Polynoms $f(x) = x^3 - e^{i \varphi} \in k[x]$. Falls $f$ das Minimalpolynom ist, ist $[k(e^{i \frac{\varphi}{3}}): k] = 3$, also $e^{i \frac{\varphi}{3}} \notin \operatorname{Kons}(M)$.
	
	Um zu sehen, dass $f \in k[x]$ tatsächlich irreduzibel ist, müssen wir $k$ verstehen!
	
	\heading{Behauptung:} $k$ ist isomorph zum Körper der rationalen Funktionen $\setQ(y)$
	
	\begin{proof}
		Ich betrachte einen Ringmorphismus
		\begin{equation*}
			\setQ[y] \to k = \setQ(e^{i \varphi}), f(y) \mapsto f(e^{i \varphi})
		\end{equation*}
		Die Funktion ist injektiv weil $e^{i \varphi}$ transzendent ist.
		
		Außerdem gilt
		\begin{equation*}
			\setQ[y] \to Q(\setQ[y]) = \setQ(y)
		\end{equation*}
		
		Die universelle Eigenschaft liefert einen Isomorphismus $\eta: \setQ(y) \to k$.
		
		$\eta$ ist surjektiv weil $e^{i \varphi} = \eta(y)$ im Bild liegt und $k$ der kleinste Körper ist, der $e^{i \varphi}$ enthält.
	\end{proof}

	Wir wollen entscheiden ob $f(x) = x^3 - e^{i \varphi} \in k[x]$ irreduzibel ist. Wir können also auch untersuchen ob $x^3 - y$ in $(\setQ(y))[x]$ irreduzibel ist.
	
	$\Leftrightarrow$ Ist $x^3 - y \in (\setQ[y])[x]$ irreduzibel?
	
	$-y$ ist prim = irreduzibel in $\setQ[y]$ und damit ist $x^3 - y$ ein Eisenstein-Polynom.
	
	\begin{example}
		Falls $p$ prim ist und das regelmäßige $p$-Eck konstruierbar ist, ist $p - 1$ von der Form $2^n$.
	\end{example}

	\begin{proof}
		Betrachte $M = \overline{M} = \{ 0, 1 \}$ und $k = \setQ(M \cup \overline{M}) = \setQ$.
		
		Das regelmäßige $p$-Eck ist konstruierbar $\Leftrightarrow$ $e^{\frac{2 \pi i}{p}} \in \operatorname{Kons}(M)$.
		
		Falls das so ist, ist
		\begin{equation*}
			\left[ \setQ(e^{\frac{2 \pi i}{p}}) : \setQ \right] = 2^n
		\end{equation*}
		für ein $n \in \setN$.
		
		Wir wissen $e^{\frac{2 \pi i}{p}}$ ist Nullstelle von $x^p - 1 \in \setQ[x]$.
		
		Aber $x^p - 1 = (x-1)(x^{p-1} + \dots + 1)$. Das Minimalpolynom ist also $x^{p-1} + \dots + 1$. Und damit $\left[ \setQ(e^{\frac{2 \pi i}{p}}) : \setQ \right] = p - 1$.
	\end{proof}

	\subsection{Ringe und Ideale}
	
	\begin{definition}
		Sei $R$ ein Ring (kommutativ, mit $1$). Sei $I \subset R$ eine nicht-leere Teilmenge. Nenne $I$ ein Ideal, falls gilt:
		\begin{enumerate}
			\item $\forall a,b, \in I: a + b \in I$
			\item $\forall a \in I, \forall r \in R: r a \in I$
		\end{enumerate}
	\end{definition}

	\begin{remark}
		Für nicht-kommutative Ringe definiert man Linksideale (wie oben) und Rechtsideale (mit $ar$ statt $ra$ in 2)).
	\end{remark}

	\begin{remark}
		\begin{itemize}
			\item Die $0$ ist in jedem Ideal enthalten
			\item $\{ 0 \}, R$ sind immer Ideale
			\item Fall $R$ ein Körper ist, sind $\{ 0 \}$ und $R$ die einzigen Ideale, denn
			
			Sei $k$ ein Körper, $I \subset k$ ein Ideal. Angenommen $\exists a \in I \setminus \{ 0 \}$. Sei $b \in k$ gegeben dann ist $b = (b \cdot a^{-1}). a \in I$.
			
			\item Falls $I \subset R$ ein Ideal und $1 \in I \Rightarrow I = R$
		\end{itemize}
	\end{remark}

	\begin{example}
		\begin{itemize}
			\item $R = \setZ, a \in \setZ$ ein Element $I = \{ \text{alle Vielfachen von $a$} \}$
			\item Besonders einfache Ideale: sei $R$ ein Ring, $I \subset R$ ein Ideal. Nenne $I$ ein Hauptideal falls $\exists a \in I: I = (a)$. Nenne $R$ Hauptidealring falls alle Ideale Hauptideale sind. z.B. $\setZ$ ist ein Hauptidealring.
			
			Sei $I \subset \setZ$ ein Ideal, $I \neq (0)$. Wir wissen: $I$ enthält positive Elemente. Sei $a \in I$ das kleinste positive Element. Will zeigen $I = (a)$. Inklusion $\supset$ ist klar. Sei also $b \in I \setminus \{ 0 \}$ irgendein Element. oBdA sei $b > 0$. Division mit Rest:
			\begin{equation*}
				\underbrace{b}_{\in I} = \underbrace{* \cdot a}_{\in I} + c, \text{ wobei } 0 \leq c < a.
			\end{equation*}
			Damit ist $c \in I$ aber auch $c < a \Rightarrow c = 0$ und damit $b \in (a)$.
			
			Das gleiche gilt falls $k$ ein Körper und $R = k[x]$ ist.
			
			$R = k[x,y]$ ist kein Hauptidealring, denn $I = (x,y)$ ist kein Hauptideal, denn
			\begin{enumerate}
				\item $I \neq R$ Genauer $1 \notin I$, denn alle Elemente von $I$ außer $0$ haben positiven Grad.
				\item Wenn $I$ ein Hauptideal wäre, $I = (a)$, dann $a \mid x$ und $a \mid y$, Aber $\operatorname{ggT}(x,y) = 1$. Also wäre $a$ Einheit, $I = R$ $\lightning$.
			\end{enumerate}
		
			Einige Rechenregeln
			\begin{itemize}
				\item $(a) \subset (b) \quad\Leftrightarrow\quad b \mid a$
				\item $(a) = (b) \quad\Leftrightarrow\quad a \sim b$
			\end{itemize}
						
			\item $R$ beliebiger Ring, $(a_\lambda)_{\lambda \in \Lambda}$ eine Familie von Elementen
			\begin{equation*}
				I = \{ r_1 \cdot a_{\lambda_1} + \dots + r_n \cdot a_{\lambda_n} \mid n \in N, r_1, \dots, r_n \in R, \lambda_1, \dots, \lambda_n \in \Lambda \}
			\end{equation*}
			
			Wir sagen das Ideal ist von $(a_\lambda)_{\lambda \in \Lambda}$ erzeugt und schreibe
			\begin{equation*}
				I = ((a_\lambda)_{\lambda \in \Lambda}) = (a_\lambda \mid \lambda \in \Lambda)
			\end{equation*}
			Falls die Familie endlich ist, schreibt man auch
			\begin{equation*}
				I = (a_1, \dots, a_n)
			\end{equation*}
		\end{itemize}
	\end{example}

	\begin{definition}
		Sei $R$ ein Ring und $I \subset R$ ein Ideal. Nenne $I$ endlich erzeugt, falls es endlich viele $a_1, \dots, a_n \in I$ gibt, so dass
		\begin{equation*}
			I = (a_1, \dots, a_n)
		\end{equation*}
	\end{definition}

	\begin{remark}
		Die Ähnlichkeit zwischen Erzeugendensystemen von Idealen und Untervektorräumen geht nicht sehr weit!
	\end{remark}

	\begin{example}
		Sei $k$ ein Körper (z.b. $\setR$) und $X \subset k^n$ eine Teilmenge (z.B. $X$ = Einheitskreis in $\setR^2$)
		
		Betrachte $R = k[x_1, \dots, x_n]$ und
		\begin{equation*}
			I = \left\{ f \in k[x_1, \dots, x_n] \mid f(x_1, \dots, x_n) = 0 \;\forall \begin{pmatrix}
				x_1 \\ \vdots \\ x_n
			\end{pmatrix} \in X \right\}
		\end{equation*}
		Diese Konstruktion ist besonders interessant, falls $X$ die Lösungsmenge eines polynomiellen Gleichungssystems ist.
	\end{example}

	\begin{definition}
		Sei $R$ ein Ring. Sage in $R$ gilt der Teilerkettensatz für Ideale, falls jede aufsteigende Kette von Idealen
		\begin{equation*}
			I_1 \subseteq I_2 \subseteq I_3 \subseteq \dots
		\end{equation*}
		nach endlich vielen Schritten konstant wird.
	\end{definition}

	\begin{theorem}
		Sei $R$ ein Ring, dann ist äquivalent
		\begin{enumerate}
			\item Jedes Ideal ist endlich erzeugt
			\item In $R$ gilt der Teilerkettensatz für Ideale
			\item In jeder nicht-leeren Menge von Idealen gibt es ein Element, das bezüglich Inklusion maximal ist
		\end{enumerate}
		Falls diese Eigenschaften gelten, nenne $R$ Noethersch
	\end{theorem}

	\begin{proof}
		\heading{1) $\Rightarrow$ 2)}
		Sei $I_1 \subseteq I_2 \subseteq \dots$ eine Folge von Idealen. Beachte:
		\begin{equation*}
			I = \bigcup_{i=0}^\infty I_i
		\end{equation*}
		ist ein Ideal, also per Annahme endlich erzeugt: $I = (a_1, \dots, a_n)$ für geeignete $a_1, \dots, a_n \in \bigcup I_i$. Dann gibt es also $i_1, \dots, i_n$ so dass $a_i \in I_{i_1}, a_2 \in I_{i_2}$ wenn $m = \max \{ i_1, \dots, i_n \}$ dann $a_1 \in I_m, a_2 \in I_m, \dots$ damit gilt:
		\begin{equation*}
			(a_1, \dots, a_n) \subset I_m \subset I = (a_1, \dots, a_m)
		\end{equation*}
		also $I_m = I_{m+1} = I_{m+2} = \dots$
		
		\heading{2) $\Rightarrow$ 3)}
		Sei $M$ eine nicht-leere Menge von Idealen ohne maximales Element. Sei $I_i \in M$ irgendein Element. Finde dann $I_2 \in M$ mit $I_1 \subsetneq I_2$. Da $I_2$ auch nicht maximal ist finde also $I_3 \in M$ mit $I_2 \subsetneq I_3$. Erhalte so eine Kette
		\begin{equation*}
			I_1 \subsetneq I_2 \subsetneq I_3 \subsetneq ...
		\end{equation*}
		$\Rightarrow$ Teilerkettensatz für Ideale gilt nicht!
		
		\heading{3) $\Rightarrow$ 1)}
		Sei $I \subset R$ ein Ideal, $I \neq (0)$. Sei $M = \{ J \subset I \mid \text{$J$ ein Ideal, $J$ endlich erzeugt} \}$
		
		Wir wissen es gibt ein maximales $m \in M$. Behauptung $m = I$
		
		Denn sonst wäre $m = (a_1, \dots, a_n) \subsetneq I$ und es gäbe $a_{n+1} \in I \setminus m$. Dann ist $m' = (a_1, \dots a_n, a_{n+1})$ endlich erzeugt, also in $M$ und $m' \supsetneq m$ $\lightning$
	\end{proof}

	\begin{theorem}[Hilbert]
		Sei $R$ Noethersch. Dann ist auch $R[x]$ Noethersch.
	\end{theorem}

	\begin{proof}
		Angenommen $R[x]$ nicht Noethersch. Wir müssen zeigen $R$ ist nicht Noethersch.
		
		Wir wissen: Es gibt in $R[x]$ ein Ideal $I$, das nicht endlich erzeugt ist.
		
		Wähle in $I$ ein Element $f$ von minimalem Grad. Dann ist $I \subsetneq (f_1)$, also $I \setminus (f_1) \neq \emptyset$, wähle $f_2 \in I \setminus (f_1)$ von minimalem Grad. $I \supsetneq (f_1, f_2)$ wähle $f_3 \in I \setminus (f_1, f_2)$ von minimalem Grad.
		
		Erhalte Folge von Polynomen $f_1, f_2, f_3, \dots$ so dass $\deg f_1 \leq \deg f_2 \leq \deg f_3 \leq \dots$
		
		Setze $n_i = \deg f_i$, $a_i = $ Leitkoeffizient von $f_i \in R$.
		
		Will zeigen, dass folgende Kette von Idealen in $R$ nicht stationär wird.
		\begin{equation*}
			(a_1) \subseteq (a_1, a_2) \subseteq (a_1, a_2, a_3) \subseteq \dots
		\end{equation*}
		dann wird klar sein, dass $R$ nicht Noethersch war.
		
		Angenommen es gäbe $k$ mit $(a_1, \dots, a_k) = (a_1, \dots, a_{k+1}) \Leftrightarrow a_{k+1} \in (a_1 \dots, a_k)$
		
		Dann gibt es also eine Linearkombination
		\begin{equation*}
			a_{k+1} = \sum_{i = 1}^k r_i a_i
		\end{equation*}
		für geeignete $r_i \in R$. Betrachte Polynom
		\begin{equation*}
			s(x) = \sum_{i = 1}^k r_i \cdot x^{n_{k+1} - n_i} \cdot f_i(x)
		\end{equation*}
		Wesentliche Eigenschaft von $s$:
		\begin{enumerate}
			\item $\deg s = n_{k+1} = \deg f_{k+1}$
			\item Leitkoeffizient $(s) = a_{k+1}$
			\item $s \in (f_1, \dots, f_k)$
		\end{enumerate}
	
		Betrachte $\underbrace{f_{k+1}(x)}_{\notin (f_1, \dots, f_k)} - \underbrace{s(x)}_{\in (f_1, \dots, f_k)} = t(x)$.
		
		Damit ist $t(x) \notin (f_1, \dots, f_k)$ und $\deg t(x) < n_{k+1}$.
		
		$\lightning$ zur Wahl von $f_{k+1}$ als Element von $I \setminus (f_1, \dots, f_k)$ von minimalem Grad.
	\end{proof}

	\begin{theorem}
		Sei $R$ ein Integritätsring, der Hauptidealring ist. Dann ist $R$ faktoriell.
	\end{theorem}

	\begin{proof}
		Sei $p$ irreduzibel, seien $a,b \in R$. $p \nmid a, p \nmid b$. Dann müssen wir zeigen: $p \nmid a \cdot b$.
		
		Wir wissen: $(p,a)$ ist ein Hauptideal, also $\exists c \in R$ so dass $(p,a) = (c)$. Also $p$ ist Vielfaches von $c$, also $c \mid p$. Aber $p$ ist irreduzibel hat also keine echten Teiler. Also $c \in R^*$ oder $c \sim p$.
		
		Aber $c \sim p \Leftrightarrow p \mid a$ was wir per Annahme ausschließen!
		
		Also $c \in R^* \quad\Rightarrow (a,p) = (1)$. Es gibt also eine Linearkombination
		\begin{equation*}
			1 = \alpha_1 a_1 + \alpha_2 p \tag{$*$}
		\end{equation*}
		Analog finde $\beta_1, \beta_2 \in R$
		\begin{equation*}
			1 = \beta_1 b + \beta_2 p \tag{$\leftmoon$}
		\end{equation*}
		
		Es folgt
		\begin{equation*}
			1 = \alpha_2 \beta_2 p^2 + (\alpha_1 \beta_2 a + \alpha_2 \beta_1 b)p + \alpha_1 a \beta_1 b
		\end{equation*}
		
		$\Rightarrow p \nmid \alpha_1 \beta_1 a b$ denn sonst würde $p$ die Summe teilen, also auch $p \mid 1$.
		
		$\Rightarrow p \nmid a \cdot b$
	\end{proof}

	\heading{Quotienten:} Sei $R$ ein Ring, $I \subset R$ ein Ideal. Dann definiere $r,s \in R$ als äquivalent, falls $r - s \in I$.
	
	\begin{theorem}
		Es gibt auf Quotientenmengen eindeutige Verknüpfungen $+, \cdot$ so dass die Quotientenabbildung
		\begin{equation*}
			q: R \to R/I
		\end{equation*}
		Ringmorphismus ist.
	\end{theorem}

	\begin{example}
		$R = \setZ, I = (p)$ das von einer Primzahl $p$ erzeugte Hauptideal. Dann gilt
		\begin{equation*}
			R/I = \setZ / p \setZ = \mathbb{F}_p = \underline{F}_p
		\end{equation*}
	\end{example}

	\begin{example}
		Sei $k$ ein Körper, $R = k[x]$, $f \in R$ ein Polynom, sowie $I = (f)$. Dann betrachte $R/(f)$.
		
		Beobachtung: Sei $n = \deg f$. Polynomdivison zeigt: die Polynome von $\deg < n$ bilden vollständiges Repräsentantensystem. Insbesondere $\dim_k R/(f) = n$.
		
		Multiplikation und Addition ist sehr einfach zu beschreiben: Wenn $a,b$ Polynome von $\deg < n$
		\begin{equation*}
			[a] \cdot [b] = [c]
		\end{equation*}
		wobei $c$ der Divisionsrest von $a \cdot b$ bei Division durch $f$ ist.
	\end{example}

	\begin{example}
		Sei $k$ ein Körper, $X \subset k^n$ eine Teilmenge (z.B. Lösungsmenge eines algebraischen Gleichungssystems).
		
		Dann setze $R = k[x_1, \dots, x_n]$
		\begin{equation*}
			I = \{ f \in R \mid f_{\mid X} \equiv 0 \}
		\end{equation*}
		und $R/I = \{ \text{Funktionen $X \to k$, die sich zu Polynomen $k^n \to k$ fortsetzen lassen} \}$ $=$ Polynomiale Funktionen $=$ algebraische Funktionen
	\end{example}

	\begin{theorem}[Universelle Eigenschaft]
		Sei $R$ ein Ring, sei $I \subset R$ ein Ideal. Sei $q: R \to R/I$ die Restklassenabbildung. Dann gilt folgende universelle Eigenschaft: für jeden surjektiven Ringmorphismus $\varphi: R \to S$ mit $\ker(\varphi) \supseteq I$ gibt es genau einen Ringmorphismus $\eta: R/I \to S$ so dass das folgende Diagramm kommutiert:
		
		\begin{center}
			\begin{tikzpicture}
				\node(R1) at (0,0){$R$};
				\node(ri) at (3,0){$R/I$};
				\node(R2) at (0,-2){$R$};
				\node(s) at (3,-2){$S$};
				
				\draw[->] (R1) -- (ri) node[midway,above]{$q$};
				\draw (R1) -- (R2) node[midway,left]{$\mathbbm{1}_R$};
				\draw[->] (R2) -- (s) node[midway,below]{$\varphi$};
				\draw[->] (ri) -- (s) node[midway,right]{$\exists ! \eta$};
			\end{tikzpicture}
		\end{center}
	\end{theorem}

	\begin{proof}""
		
		\heading{Eindeutigkeit:} Angenommen wir haben zwei Morphismen $\eta_1, \eta_2$. Sei $[a] \in R/I$ gegeben. Weil die Diagramme kommutieren, muss dann $\eta_1([a]) = \eta_1(q(a)) = \varphi(a) = \eta_2([a])$.
		
		\heading{Existenz:} Setze $\eta: R/I \to S, [a] \mapsto \varphi(a)$. Dabei ist die Wohldefiniertheit zu zeigen. Sei also $[a] = [a']$ d.h. $a - a' \in I \subset \ker(\varphi)$. Dann ist $\varphi(a) - \varphi(a') = \varphi(a - a') = 0$, also $\varphi(a) = \varphi(a')$ und die Wohldefiniertheit ist klar. Muss noch nachrechnen: $\eta$ ist Ringmorphismus, bin aber zu faul.
	\end{proof}

	\begin{example}
		Sei $\varphi: R \to S$ ein surjektiver Ringmorphismus. Dann ist $S \simeq R/\ker(\varphi)$.
	\end{example}

	\begin{proof}
		Nach universeller Eigenschaft gibt es genau eine Abbildung $\eta: R/\ker(\varphi) \to S$ so dass das folgende Diagramm kommutiert.
		
		\begin{center}
			\begin{tikzpicture}
				\node(R1) at (0,0){$R$};
				\node(ri) at (3,0){$R/\ker(\varphi)$};
				\node(R2) at (0, -2){$R$};
				\node(s) at (3,-2){$S$};
				
				\draw[->] (R1) -- (ri);
				\draw (R1) -- (R2) node[midway,left]{$\mathbbm{1}_R$};
				\draw[->] (R2) -- (s) node[midway,below]{$\varphi$};
				\draw[->] (ri) -- (s) node[midway,right]{$\exists ! \eta$};
			\end{tikzpicture}
		\end{center}
		
		Behauptung: $\eta$ ist Isomorphismus. Muss zeigen: $\eta$ bijektiv also injektiv und surjektiv. Surjektivität folgt sofort aus Kommutativität des Diagramms und der surjektivität von $\varphi$. Noch zu zeigen $\eta$ injektiv bzw. $\ker(\eta) = 0_{R/\ker(\varphi)}$.
		
		Sei also $[a] \in \ker(\eta)$. Wegen der Kommutativität des Diagramms:
		\begin{equation*}
			0_S = \eta([a]) = \eta(q(a)) = \varphi(a) \Rightarrow a \in \ker(\varphi),
		\end{equation*}
		also $[a] = 0_{R/\ker(\varphi)}$.
	\end{proof}

	\heading{Warum das Bohei um Quotienten?}
	
	Wir betrachten Körpererweiterung $L/k$ und algebraische Elemente $a \in L$.
	
	Wir wissen $a$ hat Minimalpolynom $f \in k[x]$. Jedes andere Polynom $g \in k[x]$ mit $g(a) = 0$ ist Vielfaches von $f$. ($g(a) = 0 \Leftrightarrow g \in (f)$).
	
	Betrachte Abbildung:
	\begin{align*}
		k[x] &\to k(a) \\
		g &\mapsto g(a)
	\end{align*}
	
	Wir wissen:
	\begin{itemize}
		\item $\ker(\varphi) = (f)$
		\item Die Elemente von $k(a)$ kann ich schreiben als $\lambda_1 + \lambda_2 a + \dots + \lambda_n a^{n-1}$ mit $\lambda_i \in k$ $\Rightarrow$ $\varphi$ ist surjektiv!
	\end{itemize}

	Insgesamt:
	\begin{equation*}
		k(a) \cong k[x]/(f)
	\end{equation*}

	\begin{theorem}
		Sei $\varphi: R \to S$ ein Ringmorphismus. Dann gilt
		\begin{enumerate}
			\item Für jedes Ideal $I \subset S$ ist $\varphi^{-1}(I)$ ein Ideal, das $\ker(\varphi)$ enthält.
			\item Wenn $\varphi$ surjektiv ist, dann ist die Abbildung 
			\begin{align*}
				\{ \text{Ideale in $S$} \} &\overset{\alpha}{\to} \{ \text{Ideale in $R$, die $\ker(\varphi)$ enthalten} \} \\
				I &\mapsto \varphi^{-1}(I)
			\end{align*}
			bijektiv.
			\item Wenn $\varphi$ surjektiv ist, $J \subset R$ ein Ideal, dann ist $\varphi(J) \subset S$ ein Ideal.
			\item Wenn $\varphi$ surjektiv ist, und $I \subset S$ ein Ideal ist, dann betrachte die Komposition $\psi$ von
			\begin{equation*}
				R \underset{\varphi}{\to} S \to S/I
			\end{equation*}
			und es ist $\ker(\psi) = \varphi^{-1}(I)$. Also ist $S/I \simeq R/\varphi^{-1}(I)$.
		\end{enumerate}
	\end{theorem}

	\begin{proof}
		\begin{enumerate}
			\item Hausaufgabe!
			\item Weil $\varphi$ per Annahme surjektiv ist, ist die Abbildung $\alpha$ injektiv. Also noch surjektivität zu zeigen. Sei also $J \subset R$ ein Ideal, das $\ker(\varphi)$ enthält. Wir wissen $S \simeq R/\ker(\varphi)$ also gibt es nach universeller Eigenschaft ein Diagramm
			\begin{center}
				\begin{tikzpicture}
					\node(R1) at (0,0){$R$};
					\node(ri) at (3,0){$R/\ker(\varphi)$};
					\node(R2) at (0, -2){$R$};
					\node(s) at (3, -2){$R/J$};
					
					\draw[->] (R1) -- (ri) node[midway,above]{$\varphi$};
					\draw (R1) -- (R2) node[midway,left]{$\mathbbm{1}_R$};
					\draw[->] (R2) -- (s) node[midway,below]{$q$};
					\draw[->] (ri) -- (s) node[midway,right]{$\exists ! \eta$};
				\end{tikzpicture}
			\end{center}
			und $J = q^{-1}((0)) = \varphi^{-1}(\eta^{-1}(0))$, setze $I = \eta^{-1}(0)$, fertig.
			\item Sei $J \subset R$ ein Ideal. Muss zeigen
			
			\heading{C 1}: Wenn $a,b \in \varphi(J)$, dann ist $a + b \in \varphi(J)$. $\exists a', b' \in J$ mit $a =  \varphi(a'), b = \varphi(b')$ und dann $a + b = \varphi(\underbrace{a' + b'}_{\in J})$
			
			\heading{C2:} Sei $a \in \varphi(J)$, sei $b \in S$ beliebig. Dann ist $s \cdot a \in \varphi(J)$. Weil $\varphi$ surjektiv ist, $\exists s' \in R: s = \varphi(s')$. Außerdem $\exists a' \in J: a = \varphi(a')$ und $\varphi(\underbrace{s' a'}_{\in J}) = \varphi(s') \varphi(a') = sa$
			
			\item Sei $r \in R$. Es gilt
			\begin{align*}
				r \in \ker(\psi) &\Leftrightarrow q(\varphi(r)) = 0_{S/I} \\
				&\Leftrightarrow \varphi(r) \in I \\
				&\Leftrightarrow r \in \varphi^{-1}(I)
			\end{align*}
		\end{enumerate}
	\end{proof}

	\begin{corollary}
		Sei $R$ noethersch (bzw. Hauptidealring). Sei $I \subset R$ ein Ideal. Dann ist $R/I$ Noethersch (bzw. Hauptidealring).
	\end{corollary}

	\heading{Notation:} Sei $R$ Ring. Seien $I \subseteq J \subseteq R$ Ideale. Dann betrachte $q_I: R \to R/I$.
	
	Das Ideal $q_I(J) \subseteq R/I$ wird mit $J/I$ bezeichnet.

	\begin{theorem}[Noetherscher Isomorphiesatz]
		Situation wie oben. Dann
		\begin{equation*}
			R/J \simeq (R/I) / (J/I)
		\end{equation*}
	\end{theorem}

	\begin{proof}
		Wir haben Ringmorphismen
		\begin{equation*}
			R \overset{q_I}{\longrightarrow} R/I \overset{q_{J/I}}{\longrightarrow} (R/I) / (J/I)
		\end{equation*}
		Wir wissen $\ker(\eta) = q_I^{-1}(J/I) = J$. Also folgt die Aussage.
	\end{proof}

	\heading{Haben 2 wichtige Typen von Idealen}
	\begin{itemize}
		\item Primideale: $R$ ein Ring, $I \subseteq R$ ein Ideal. Nenne $I$ prim, falls $\forall a,b \in R: a \cdot b \in I \Rightarrow a \in I \lor b \in I$
		\item Maximale Ideale: $R$ ein Ring. Ein Ideal $I \subset R$ heißt maximal falls gilt
		\begin{enumerate}
			\item $I \neq R$
			\item Wenn $J \supsetneq I$ ein echt größeres Ideal ist, dann ist $J = R$.
		\end{enumerate}
	\end{itemize}

	\begin{example}
		\begin{itemize}
			\item Sei $R$ ein Ring, $p \in R$ ein prim-Element. Dann ist $(p)$ ein Primideal.
			\item Sei $k$ ein Körper, $R = k[x_1, \dots, x_n]$ und
			\begin{equation*}
				I = (x_1, x_2, \dots, x_n)= \{ \underbrace{x_1 f_1 + x_2 f_2 + \cdots + x_n f_n}_\text{Haben stets Nullstelle am Ursprung!} \mid f_i \in k[x_1, \dots, x_n] \}
			\end{equation*}
			Wir wissen $1 \notin I$, denn $1$ hat keine Nullstelle.
			
			Beobachte: Ein Polynom liegt genau dann in $I$ wenn der konstante Teil gleich Null ist (d.h. wenn $f(0) = 0_k$).
			
			Sei jetzt $J \supsetneq I$ echt größer! Sei $f \in J \setminus I$ Dann
			\begin{equation*}
				\underbrace{f}_{\in J} = \text{const}^{\neq 0} + (\underbrace{\text{Polynom ohne konstanten Teil}}_{\in I \subset J})
			\end{equation*}
			$\Rightarrow \text{const}^{\neq 0} \in J \Rightarrow J = R$
			
			Variante: Seien $a_1, \dots, a_n \in k$. Dann ist $I' = (x_1 - a_1, x_2 - a_2, \dots, x_n - a_n)$ auch maximal.
			
			Zurück zu Beispiel ohne Variante
			\begin{align*}
				R/I = k[x_1, \dots, x_n]/(x_1, \dots, x_n) &\overset{\simeq}{\longrightarrow} k \\
				[f] &\longmapsto f(0)
			\end{align*}
		\end{itemize}
	\end{example}

	\begin{example}
		Sei $k$ ein Körper, $f \in k[x]$ irreduzibel. Dann ist $(f)$ maximal.
	\end{example}

	\begin{proof}
		Sei $J \supsetneq (f)$ größer, sei $g \in J \setminus (f)$ ein Element ($g$ kein Vielfaches von $f$).
		
		Wissen (Euklidischer Algorithmus): $\operatorname{ggT}(f,g) \in J$. Aber $f$ ist irreduzibel hat also keine echten Teiler d.h. $\operatorname{ggT}(f,g) = 1$
	\end{proof}

	\begin{theorem}
		Sei $R$ ein Ring, $I \subset R$ ein Ideal. Dann gilt
		\begin{enumerate}
			\item $I$ ist prim $\Leftrightarrow$ $R/I$ ist Integritätsring
			\item $I$ ist maximal $\Leftrightarrow$ $R/I$ ist ein Körper
		\end{enumerate}
		Insbesonders maximale Ideale sind prim (denn Körper sind Integritätsringe)
	\end{theorem}

	\begin{proof}
		\heading{$1) \Rightarrow$:} Sei $I$ prim. Seien $[a],[b] \in R/I$ Äquivalenzklassen von Elementen $a,b \in R$ so dass $[a] \neq 0_{R/I}$ und $[b] \neq 0_{R/I}$. Dann gilt $a \notin I$ und $b \notin I$.
		
		Da $I$ prim $a \cdot b \notin I \Rightarrow [a \cdot b] \neq 0_{R/I}$
		
		\heading{$1) \Leftarrow$:} Sei $R/I$ ein Integritätsring. Seien $a,b \in R \setminus I$. Dann $[a] \neq 0_{R/I}$ und $[b] \neq 0_{R/I}$ und $[a \cdot b] \neq 0_{R/I}$.
		
		$\Rightarrow a b \notin I$
		
		\heading{$2) \Rightarrow$:} Sei $I$ maximal. Sei $a \in R$ mit $[a] \neq 0_{R/I}$ d.h. $a \notin I$.
		
		Dann betrachte $J = (I, a)$. Wir wissen $J \supsetneq I$ also $(1) = J$. Also kann ich schreiben:
		\begin{equation*}
			1 = f + g \cdot a \qquad \text{mit $f \in I, g \in R$}
		\end{equation*}
		\begin{equation*}
			\Rightarrow \underbrace{[1]}_{=1_{R/I}} = \underbrace{[f]}_{0_{R/I}} + [g] \cdot [a]
		\end{equation*}
		also ist $[g] = [a]^{-1}$ in $R/I$
		
		\heading{$2) \Leftarrow$:} Sei $R/I$ ein Körper, sei $J \supsetneq I$ ein echtes Oberideal. Dann gibt es $a \in J \setminus I$.
		
		Wir wissen $[a] \neq 0_{R/I}$, per Annahme $\exists b \in R$ mit $[a] \cdot [b] = [1]$. Das bedeutet $\exists f \in I$ so dass
		\begin{equation*}
			\underbrace{a \cdot b}_{\in J} + \underbrace{f}_{\in I \subset J} = 1
		\end{equation*}
		das heißt $1 \in J$ d.h. $J = R$.
	\end{proof}

	\begin{remark}
		Teil 2) des Satzes gibt neuartige Methode, um Beispiele von Körpern zu konstruieren!
	\end{remark}

	\heading{Weitere Beobachtungen/Konstruktionen mit Idealen}
	
	Sei $R$ ein Ring, seien $I_1, \dots, I_n$ Ideale in $R$
	\begin{itemize}
		\item Dann ist $I_1 \cap I_2 \cap \dots \cap I_n$ ein Ideal
		\item Dann ist $I_1 + \dots + I_n = \{ f_1 + \dots + f_n \in R \mid \forall i f_i \in I_i \}$ ein Ideal
	\end{itemize}

	\begin{example}
		$R = \setZ$ $I_1 = (a)$ $I_2 = (b)$
		
		$I_1 \cap I_2 = (\operatorname{kgV}(a,b))$
		
		$I_1 + I_2 = (\operatorname{ggT}(a,b))$
	\end{example}

	\begin{definition}
		Zwei Ideale $I_1, I_2$ heißen Teilerfremd, wenn $I_1 + I_2 = (1)$.
	\end{definition}

	\heading{Nutzanwendung:} Manchmal hat man Aufgaben der Form: egeben Ring $R$ gegeben Ideale $I_1, \dots, I_n$ und Elemente $r_1, \dots, r_n \in R$. Finde ein/alle $r \in R$
	\begin{align*}
		r &\equiv r_1 \mod I_1 \\
		r &\equiv r_2 \mod I_2 \\
		&\vdots \\
		r &\equiv r_n \mod I_n
	\end{align*}
	
	\heading{Antwort ist Chinesischer Restsatz:} Situation wie oben. Fall $\forall \; i \neq j$ die Ideale $I_i$ und $I_j$ stets teilerfremd sind, dann ist die Abbildung:
	\begin{align*}
		\varphi: R &\to R/I_1 \times R/I_2 \times \dots \times R/I_n \\
		r &\mapsto ([r]_{R/I_1}, [r]_{R/I_2}, \dots, [r]_{R/I_n})
	\end{align*}
	surjektiv und $\ker(\varphi) = I_1 \cap \dots \cap I_n$.
	
	\begin{proof}
		Aussage über $\ker(\varphi)$ ist trivial. Müssen surjektiv zeigen!
		
		Seien $k \neq l$ gegeben. Wir wissen $(1) = I_k + I_l$. Also existieren Elemente $a_{kl} \in I_k$ und $b_{kl} \in I_l$ so dass $1 = a_{kl} + b_{kl}$
		
		Setze
		\begin{equation*}
			s_l = \prod_{k \neq l} a_{kl} = \prod_{k \neq l} (1 - b_{kl}) \in R
		\end{equation*}
		
		\textit{Beobachtung:} Seien $k \neq l$ gegeben. Dann $s_l \equiv 0 \mod I_k$ denn Faktor $a_{kl}$ aus dem 1. Produkt ist $\equiv 0 \mod I_k$.
		
		$s_l \equiv 1 \mod I_l$, denn es ist stets $b_kl \equiv 0 \mod I_l$, also jeder Faktor des Rechen Produktes $\equiv 1 \mod I_l$.
		
		Seien $r_1, \dots, r_n \in R$ gegeben.
		
		Setze: $r = \sum r_i \cdot s_i$ dann gilt $\forall i: r \equiv r_i \mod I_i$, also
		\begin{equation*}
			\varphi(r) = [r_1] \times [r_2] \times \dots \times [r_n]
		\end{equation*}
	\end{proof}

	\heading{Einschub Mengenlehre}
	
	\begin{definition}
		Sei $M$ eine Menge. $\leq$ sei eine Relation. Ich nenne $\leq$ eine Halbordnung falls gilt:
		\begin{enumerate}
			\item $\forall a \in M: a \leq a$
			\item Wenn $a,b,c \in M$ gegeben sind mit
			\begin{equation*}
				a \leq b, b \leq c \Rightarrow a \leq c
			\end{equation*}
			\item $\forall a,b \in M: a \leq b$ und $b \leq a \Rightarrow a = b$
		\end{enumerate}
		Wir fordern nicht dass $\forall a,b \in M: a \leq b$ oder $b \leq a$ gilt. (Falls das gilt nenne $\leq$ vollständig)
	\end{definition}

	\begin{example}
		Betrachte $S = \text{Studierende}$, $M = \operatorname{Pot}(S)$.
		
		Gegeben $m_1, m_2 \in M$, schreibe $m_1 \leq m_2$ falls $m_1 \subseteq m_2$ ist.
	\end{example}

	\begin{definition}
		Sei $(M, \leq)$ eine Mengen mit Halbordnung. Eine Kette ist eine Teilmenge $N \subset M$, so dass die auf $N$ induzierte Halbordnung vollständig ist. Ein Element $m \in M$ heißt obere Schranke der Kette $N$, falls $\forall a \in N: n \leq m$.
	\end{definition}

	\begin{example}
		Sei $(M, \leq)$ gegeben. Sei $(n_i)_{i \in \setN}$ eine Folge von Elementen so dass $n_1 \leq n_2 \leq \dots$ ist. Dann ist $N = \{ n_i \mid i \in \setN \}$ eine Kette.
	\end{example}

	\begin{example}
		Sei $M = \setR$ und $\leq$ wie üblich definiert. Dann ist jede Teilmengen eine Kette, denn $\leq$ ist sowieso vollständig. Obere Schranken existieren genau dann wenn $N$ nach oben beschränkt ist.
	\end{example}

	\begin{theorem}[Lemma von Zorn]
		Sei $(M, \leq)$ eine halbgeordnete Megne, $M \neq \emptyset$. Falls jede Kette eine obere Schranke besitzt, dann gibt es in $M$ ein maximales Element.
	\end{theorem}

	\begin{remark}
		Dies ist Äquivalent zum Auswahlaxiom. Sei $(M_\alpha)_{\alpha \in A}$ eine Familie von Mengen. Dann gibt es eine Abbildung
		\begin{equation*}
			A \to \bigcup_{\alpha \in A} M_\alpha
		\end{equation*}
		so dass $\forall \alpha \in A: \varphi(\alpha) \in M_\alpha$.
	\end{remark}

	\begin{theorem}
		Sei $R$ ein Ring. $I \subset R$ ein Ideal. Dann gibt es ein maximales Ideal $m \subset R$, das $I$ enthält
	\end{theorem}
	
	\begin{proof}
		Sei
		\begin{equation*}
			M = \{ \text{Ideale $J \subset R$ mit $I \subseteq J \subsetneq R$} \}
		\end{equation*}
		wähle $\subseteq$ als Halbordnung.
		
		Beachte: Wenn $N \subset M$ eine Kette ist, dann ist $s = \bigcup_{n \in N} n$ eine obere Schranke.
		\begin{itemize}
			\item Ketteneingenschaft garantiert, dass $s$ ein Ideal ist
			\item $1 \notin s$, denn für alle $m \in M: 1 \notin m$. Also $s \subsetneq R$, also $s \in M$
		\end{itemize}
		Zorn: Es existiert in $M$ ein maximales Element $m$.
		
		Nachrechnen: Dies ist ein maximales Ideal in $R$, welches $I$ enthält.
	\end{proof}
	
	\section{Körpertheorie}
	
	\subsection{Grundbegriffe}
	
	\heading{Beobachtung:} Sei $k$ ein Körper sei $1_k$ das neutrale Element der Multiplikation. Dann betrachte Ringmorphismus
	\begin{align*}
		\eta: \setZ &\to k \\
		n &\mapsto \begin{cases}
			\underbrace{1_k + \dots + 1_k}_\text{$n$ mal} & \text{falls } n \geq 0 \\
			\underbrace{-(1_k + \dots + 1_k)}_\text{$-n$ mal} & \text{falls } n < 0 \\
		\end{cases}
	\end{align*}
	
	\heading{Beobachte:} Wenn $k' \subset k$ ein Unterkörper ist, dann $\operatorname{Bild}(\eta) \subseteq k'$.
	
	\heading{Beobachtung:} Wenn $(k_\lambda)_{\lambda \in \Lambda}$ eine Familie von Unterkörpern ist, dann ist
	\begin{equation*}
		k' \coloneqq \bigcap_{\lambda \in \Lambda} k_\lambda
	\end{equation*}
	wieder ein Unterkörper.
	
	\begin{definition}
		Gegeben ein Körper $k$ betrachte
		\begin{equation*}
			k' \coloneqq \bigcap_{\substack{k'' \subseteq k \\ \text{Unterkörper}}} k''
		\end{equation*}
		Dieser Unterkörper heißt Primkörper von $k$.
	\end{definition}

	\heading{Mit der Beobachtung von eben:} $\operatorname{Bild}(\eta) \subseteq \text{Primkörper}$
	
	\heading{Beachte:} $\eta$ ist entweder injektiv oder nicht.
	
	\heading{Fall $\eta$ ist injektiv:}
	
	\begin{center}
		\begin{tikzpicture}
			\node(R1) at (0,0){$\setZ$};
			\node(ri) at (5,0){$Q(\setZ) = \setQ$};
			\node(R2) at (0, -2){$\setZ$};
			\node(s) at (5, -2){Primkörper von $k$};
			
			\draw[->] (R1) -- (ri) node[midway,above]{$\varphi$};
			\draw (R1) -- (R2) node[midway,left]{$\mathbbm{1}_R$};
			\draw[right hook->] (R2) -- (s) node[midway,below]{$\eta$};
			\draw[right hook->] (ri) -- (s) node[midway,right]{$\exists ! \varphi$};
		\end{tikzpicture}
	\end{center}

	\heading{Beachte:} $\operatorname{Bild}(\varphi)$ ist Unterkörper des Primkörpers, welcher der kleinste Unterkörper von $k$ ist, also $\operatorname{Bild}(\varphi) = \text{Primkörper}$. Also insgesamt: Falls $\eta$ injektiv ist ist der Primkörper kanonisch isomorph zu $\setQ$.
	
	\heading{Fall $\eta$ nicht injektiv:} Dann ist $\ker(\eta) \subseteq \setZ$ ein nicht-triviales Ideal.
	
	Weil $\eta(1_\setZ) = 1_k \neq 0_k$, ist $\ker(\eta) \subsetneq \setZ$ also Hauptideal der Form $(p)$ für ein $p \in \setN$. Weil $k$ nullteilerfrei ist, ist $p$ eine Primzahl und nach universeller Eigenschaft von Quotienten habe ich ein Diagramm.
	
	\begin{center}
		\begin{tikzpicture}
			\node(R1) at (0,0){$\setZ$};
			\node(ri) at (5,0){$\setZ/(p)$};
			\node(R2) at (0, -2){$\setZ$};
			\node(s) at (5, -2){Primkörper von $k$};
			
			\draw[->] (R1) -- (ri) node[midway,above]{$\varphi$};
			\draw (R1) -- (R2) node[midway,left]{$\mathbbm{1}_R$};
			\draw[->] (R2) -- (s) node[midway,below]{$\eta$};
			\draw[right hook->] (ri) -- (s) node[midway,right]{$\exists \varphi$};
		\end{tikzpicture}
	\end{center}

	Argumentiere wie oben, erhalte einen kanonischen Isomorphismus zwischen dem Primkörper und $\setZ / p \setZ$.
	
	\heading{Zusammenfassung/Notation:} Sei $k$ ein Körper. Sei $k' \subseteq k$ der Primkörper. Dann entweder
	\begin{itemize}
		\item $k' \simeq \setQ$ und man sagt: $k$ hat Charakteristik $0, \operatorname{char}(k) = 0$
		\item $k' \simeq \setZ / p \setZ$ für eine Primzahl $p$ und man sagt $k$ hat die Charakteristik $p$
	\end{itemize}

	\heading{Bemerkung zum Gruseln:} Sei $\operatorname{char}(k) = p > 0$. Dann ist $(x + y)^p = x^p + y^p$. Insbesondere ist
	\begin{align*}
		\operatorname{Frob:} k[x] &\to k[x] \\
		f &\mapsto f^p
	\end{align*}
	ein Ringmorphismus. Außerdem ist die Ableitung von $f(x) = x^p$ gegeben als $f'(x) = p x^{p-1} \equiv 0$.
	
	$f(x) = x^p + x^{p+2}$ und $f'(x) = (p + 2) \cdot x^{p+1} = 2 \cdot x^{p+1}$
	
	\heading{Schlussbeobachtung:} Sei $k$ ein endlicher Körper, dann ist $\operatorname{char}(k) = p > 0$. Beobachte: $k$ ist ein Vektorraum über dem Primkörper $\simeq \setZ / p \setZ$.
	
	Sei $n = \dim_\text{Prim} k$. Dann $n < \infty$ und $\#k = p^n$.
	
	\subsection{Der algebraische Abschluss}
	
	\heading{Beobachtung:} Das Polynom $x^2 + 2$ hat in $\setQ$ keine Nullstelle, aber im Oberkörper $\setC$. Es gilt sogar jedes nicht konstante $f \in \setC[x]$ hat in $\setC$ eine Nullstelle.
	
	\heading{Ziel:} Wir wollen ähnliches für beliebige Körper konstruieren. Gegeben Körper $k$. Konstruiere einen Oberkörper $\overline{k}$ so dass alle nicht konstanten Polynome $f \in \overline{k}[x]$ in $\overline{k}$ eine Nullstelle haben.
	
	\heading{Aber:} $\overline{k}$ erfüllt keine gute universelle Eigenschaft $\leadsto$ Galois-Theorie: Symmetrie von Erweiterungen
	
	\heading{Spielwiese:} Betrachte $\setQ$ und $k = \setQ[x]/(x^2 + 1)$.
	
	Ich kann $\setQ$ in $k$ einbetten durch
	\begin{align*}
		\setQ &\hookrightarrow k \\
		q &\mapsto [q]
	\end{align*}
	Also $k$ ist Oberkörper von $\setQ$.
	
	Betrachte das Element $a \coloneqq [x] \in k$
	
	\heading{Beobachte:} $a^2 + 1_k = a \cdot a + 1_k = [x] [x] + [1_\setQ] = [x \cdot x + 1_\setQ] = [x^2 + 1_\setQ] = 0_k$.
	
	\heading{Einsicht:} $a \in k$ ist Nullstelle des Polynoms $x^2 + 1_k \in k[x]$
	
	Wie soll die Konstruktion von $\overline{k}$ gehen? Grundidee: so wie in der Spielwiese.
	
	\begin{theorem}
		Sei $k$ ein Körper, sei $f \in k[x]$ nicht konstant. Dann gibt es einen Oberkörper $L \supseteq k$ so dass $f$ als Polynom in $L[x]$ eine Nullstelle in $L$ hat.
	\end{theorem}

	\begin{proof}
		Sei $p(x)$ ein irreduzibler Faktor von $f$. Setze
		\begin{equation*}
			L \coloneqq k[x] / (p)
		\end{equation*}
		das ist ein Körper, weil $(p)$ maximales Ideal ist.
		
		Bette $k$ mit Hilfe des injektiven Körpermorphismus
		\begin{align*}
			k &\to L \\
			a &\mapsto [a]
		\end{align*}
		in $L$ ein. Beachte, dass $a \coloneqq [x] \in L$ eine Nullstelle von $p$ und also auch von $f$ ist.
	\end{proof}

	\heading{Beobachtung:} Wir wissen schon: wenn ich diese Konstruktion anwende auf $k = \setR, f = x^2 + 1$ dann erhalte ich $\setC$. Ich sehe schon an diesem Beispiel, dass die so erhaltene Erweiterung Symmetrien besitzt, nämlich die komplexe Konjugation. Also ist es nicht richtig, dass $\setC$ eindeutig ist bis auf kanonische Isomorphie.
	
	\begin{theorem}
		Sei $k$ ein Körper. Dann ist äquivalent:
		\begin{enumerate}
			\item Jedes nicht-konstante Polynom in $k[x]$ hat Nullstelle in $k$.
			\item Jedes nicht-konstante Polynom zerfällt in Linearfaktoren
			\item Jedes irreduzible Polynom ist linear
			\item Wenn $L/k$ eine algebraische Körpererweiterung ist, dann ist $L = k$
		\end{enumerate}
		Nenne $k$ algebraisch abgeschlossen falls diese Bedingungen erfüllt sind.
	\end{theorem}

	\begin{proof}
		\heading{$1) \Rightarrow 2)$:} Polynomdivision: wenn $f$ bei $a$ eine Nullstelle hat dann ist $f$ ein Vielfaches von $(x - a)$.
		
		\heading{$2) \Rightarrow 3)$:} trivial
		
		\heading{$3) \Rightarrow 4)$:} Sei $L/k$ eine algebraische Körpererweiterung. Sei $a \in L$ gegeben. Dann $a$ ist algebraisch über $k$. Sei $f \in k[x]$ das Minimalpolynom. Dann $f$ irreduzibel, also linear, also $f(x) = x - a \in k[x]$ $\Rightarrow a \in k$.
		
		\heading{$4) \Rightarrow 1)$:} Sei $f \in k[x]$ nicht konstant. Sei $p(x)$ ein irreduzibler Faktor von $f$. Setze
		\begin{equation*}
			L = k[x]/(p)
		\end{equation*}
		das ist eine endliche Erweiterung, denn $\dim_k L = \deg p < \infty$, also ist $L$ algebraisch.  Außerdem gilt: $f$ hat in $L$ eine Nullstelle. Nach 4) ist $L = k$ also hat $f$ bereits in $k$ eine Nullstelle.
		
		\begin{definition}
			Sei $k$ ein Körper. Ein Oberkörper $\overline{k}/k$ heißt algebraischer Abschluss von $k$ falls gilt:
			\begin{enumerate}
				\item $\overline{k}$ ist algebraisch abgeschlossen
				\item $\overline{k} / k$ ist algebraisch
			\end{enumerate}
		\end{definition}
	
		\heading{Achtung:} $\setC$ ist kein algebraischer Abschluss von $\setQ$!
		
		\heading{Nicht verwechseln} mit algebraischer Abschluss von $k$ in einem Oberkörper $L$ $= \{ l \in L \mid \text{$l$ ist algebraisch über $k$} \}$. 
	\end{proof}

	\begin{definition}
		Seien $R,S$ Ringe (später meistens Körper) die beide den Ring $T$ als Unterring besitzen.
		
		Ein Ringmorphismus $\varphi: R \to S$ heißt $T$-Morphismus, falls $\varphi \mid_T = id_T$.
	\end{definition}

	\begin{example}
		$R = S = \setC$, $T = \setR$. Dann ist die Konjugation
		\begin{align*}
			\varphi: \setC &\to \setC \\
			z &\mapsto \overline{z}
		\end{align*}
		ein $\setR$-Morphismus.
	\end{example}

	\begin{theorem}
		\label{thm:schwacherErsatzFürUniverselleEigenschaft}
		Sei $k$ ein Körper, $\overline{k}$ ein algebraischer Abschluss von $k$. Sei $L/k$ algebraisch, sei $L_0$ ein Zwischenkörper $k \subseteq L_0 \subseteq L$. Sei weiter ein $k$-Morphismus $\varphi_0: L_0 \to \overline{k}$ gegeben. Dann existiert eine Fortsetzung $\varphi: L \to \overline{k}$ (d.h. ein Körpermorphismus $\varphi$, so dass $\varphi \mid_T{L_0} = \varphi_0$).
		
		Insbesondere $(L_0 = k)$ jede algebraische Körpererweiterung von $k$ bettet in $\overline{k}$ ein.
	\end{theorem}

	\heading{Typische Anwendung:} Sei $k$ ein Körper, seien $\overline{k}$ und $\overline{k}'$ zwei algebraische Abschlüsse von $k$. Dann $\overline{k} \simeq \overline{k}'$.
	
	\begin{proof}
		Wende den Satz \ref{thm:schwacherErsatzFürUniverselleEigenschaft} an mit $L = \overline{k}', L_0 = k$ und $\varphi_0 = Id_k$. Satz sagt dann, es gibt Körpermorphismus (sogar $k$-Morphismus)
		\begin{equation*}
			\varphi: \overline{k}' \to \overline{k}
		\end{equation*}
		Wissen: $\varphi$ ist injektiv. Ich behaupte: sogar surjektiv. Grund: Haben Kette von Körpern $k \subseteq \operatorname{Bild}(\varphi) \subseteq \overline{k}$.
		
		Wissen $\operatorname{Bild}(\varphi) \simeq \overline{k}'$ ist algebraisch abgeschlossen. $\overline{k}/k$ ist algebraisch $\Rightarrow$ $\overline{k} / \operatorname{Bild}(\varphi)$ ist algebraisch.
		
		Insgesamt: $\overline{k} = \operatorname{Bild}(\varphi)$, denn algebraisch abgeschlossene Körper haben keine echten algebraischen Erweiterungen.
	\end{proof}

	\begin{proof} zu Satz \ref{thm:schwacherErsatzFürUniverselleEigenschaft}
		Verwende Zorns Lemma und betrachte
		\begin{align*}
			M = \{ (L', \varphi') \mid &L' \text{ ist Zwischenkörper } L_0 \subseteq L' \subseteq L \text{ und } \\
			&\varphi': L' \to \overline{k} \text{ ist Körpermorphismus mit } \varphi'_{\mid L_0} = \varphi_0 \}
		\end{align*}
		
		Definiere eine Halbordnung durch $(L', \varphi') \leq (L'', \varphi'')$ falls gilt:
		\begin{enumerate}
			\item $L' \subseteq L''$
			\item $\varphi''_{\mid L'} = \varphi'$
		\end{enumerate}
		Fakt ohne Beweis: Das ist tatsächlich eine Halbordnung.
		
		\textit{Zwischenbehauptung:} In $(M, \leq)$ hat jede Kette eine obere Schranke.
		
		Sei $(L_\lambda, \varphi_\lambda)_{\lambda \in \Lambda}$ eine Kette. Dann ist $L' \coloneqq \bigcup_{\lambda \in \Lambda} L_\lambda$ ein Unterkörper von $L$ (sogar Zwischenkörper: $L_0 \subseteq L' \subseteq L$). Sei $a \in L'$ und seien $\lambda_1, \lambda_2 \in \Lambda$ so dass $a \in L_{\lambda_1}$ und $a \in L_{\lambda_2}$ ist. Dann gilt:
		\begin{equation*}
			\varphi_{\lambda_1}(a) = \varphi_{\lambda_2}(a)
		\end{equation*}
		Auswahlaxiom sagt: finde Abbildung $\eta: L' \to \Lambda$ so dass für alle $a$ $L_{\eta(a)} \ni a$.
		
		Definiere dann:
		\begin{align*}
			\varphi': &L' \to \overline{k} \\
			&a \to \varphi_{\eta(a)}(a)
		\end{align*}
		
		Das ist ein Körpermorphismus, der $\varphi_0$ fortsetzt. Also ist $(L', \varphi')$ eine obere Schranke für die Kette.
		
		Insgesamt sagt Zorns Lemma: Es gibt ein maximales Element $(L_\text{max}, \varphi_\text{max}) \in M$. Ich bin fertig, wenn ich zeige:
		$L_\text{max} = L$. Angenommen es gibt $a \in L \setminus L_\text{max}$.
		
		\textit{Wissen:} $a$ ist algebraisch über $L_\text{max}$, mit Minimalpolynom
		\begin{equation*}
			f(x) = \sum \lambda_i x^i \in L_\text{max}[x]
		\end{equation*}
		
		\textit{Wissen auch:}
		\begin{equation*}
			L_\text{max}(a) \simeq L_\text{max}[x] / (f)
		\end{equation*}
		
		Betrachte das Polynom
		\begin{equation*}
			\overline{f} = \sum \varphi_\text{max}(\lambda_i) \cdot x^i \in \operatorname{Bild}(\varphi_\text{max})[x] \subset \overline{k}[x]
		\end{equation*}
		
		\textit{Wissen:} $\overline{f}$ hat eine Nullstelle $\overline{a} \in \overline{k}$ und 
		\begin{equation*}
			\operatorname{Bild}(\varphi_\text{max})(\overline{a}) \simeq \operatorname{Bild}(\varphi_\text{max})[x] / (\overline{f}) \simeq L_\text{max}[x]/ (f) \simeq L_\text{max}(a)
		\end{equation*}
		Insgesamt haben wir also einen Morphismus
		\begin{equation*}
			L_\text{max} \subsetneq L_\text{max}(a) \overset{\varphi_\text{mmax}}{\hookrightarrow} \operatorname{Bild}(\varphi_{\eta(a)}\text{max})(\overline{a}) \subseteq \overline{k}
		\end{equation*}
		
		Per Konstruktion ist $\varphi_\text{mmax} \mid_{L_\text{max}} = \varphi_\text{max}$
		
		Insgesamt: $(L_\text{max}, \varphi_\text{max}) \lneq (L_\text{max}(a), \varphi_\text{mmax})$ $\lightning$ zur Maximalität von $(L_\text{max}, \varphi_\text{max})$.
	\end{proof}

	\begin{definition}[Polynomringe in $\infty$ vielen Variablen]
		Sei $(x_\lambda)_{\lambda \in \Lambda}$ eine Menge von Variablennamen, sei $R$ ein Ring Dann betrachte
		\begin{equation*}
			R[(x_\lambda)_{\lambda \in \Lambda}] = \bigcup_{ \{ x_{\lambda_1}, \dots, x_{\lambda_n} \} \text{ endl.} } R[x_{\lambda_1}, \dots, x_{\lambda_n}]
		\end{equation*}
	\end{definition}

	\begin{remark}
		Polynome enthalten immer nur endlich viele Terme und endlich viele Variablen!
	\end{remark}

	\heading{Fakt:} (universelle Eigenschaft) Gegeben sei ein Ringmorphismus $\varphi: R \to S$ und eine beliebige Abbildung: $\alpha: \Lambda \to S$. Dann ibt es genau einen Ringmorphismus $\Phi: R[(x_\lambda)_{\lambda \in \Lambda}] \to S$ so dass $\Phi_{\mid R} = \varphi$
	\begin{equation*}
		\exists \lambda \in \Lambda: \Phi(x_\lambda) = \alpha(\lambda)
	\end{equation*}
	\textit{Idee:}
	\begin{equation*}
		\Phi(x_{\lambda_1}^2 + x_{\lambda_2} + r \cdot x_{\lambda_3}^7 \cdot x_{\lambda_4}) = \alpha(\lambda_1)^2 + \alpha(\lambda_2) + \varphi(r) \cdot \alpha(\lambda_3)^7 \cdot \alpha(\lambda_4)
	\end{equation*}

	\begin{theorem}[Steinitz]
		Sei $k$ ein Körper. Dann existiert ein algebraischer Abschluss.
	\end{theorem}

	\begin{proof}(Mike Artin)
		Betrachte:
		\begin{itemize}
			\item $\Lambda = \{ \text{nicht-konstante Polynome in $k[x]$} \}$
			\item Polynomring $k[(x_\lambda)_{\lambda \in \Lambda}] \eqqcolon P$
			\item Für jedes $f \in \Lambda$ das Element $f(x_f)$
			\item Das Ideal $I = (f(x_f) \mid f \in \Lambda)$
		\end{itemize}
		\textit{Behauptung 1:} $I \subsetneq P$ d.h. $1 \notin I$
		
		\textit{Beweis:} Angenommen es wäre $1 \in I$. Dann kann ich schreiben
		\begin{equation*}
			1 = \sum_{i=1}^n g_i \cdot f_i(x_{f_i})
		\end{equation*}
		für geeignete $f_1, \dots, f_n \in \Lambda, g_1, \dots, g_n \in P$. Das kann nicht sein!
		
		\textit{Erinnerung:} Es gibt eine Körpererweiterung $k_1 / k$ so dass $f_1$ eine Nullstelle $a_1 \in k_1$ hat.
		
		Wiederholte Anwendung: Es gibt eine Körpererweiterung $k' /  k$ so dass für alle $i$ gilt $f_i$ hat in $k'$ eine Nullstelle $a_i \in k'$.
		
		\textit{Universelle Eigenschaft:} Es gibt Ringmorphismus $\Phi: P \to k'$ so dass für alle $i$ gilt $x_{f_i} \mapsto a_i$.
		
		Dann ist
		\begin{equation*}
			\Phi(1_{k'}) = \Phi(1_P) = \sum_{i = 1}^n \underbrace{\Phi(g_i) f_i(a_i)}_{=0} = 0
		\end{equation*}
		Widerspruch! Damit ist Behauptung 1 bewiesen.
		
		\textit{Erinnerung:} $I$ ist vielleicht nicht maximal, aber Zorn sagt: Es gibt ein maximales Ideal $I \subseteq m \subsetneq P$.
		
		\textit{Erinnerung:} $E_1 \coloneqq P/m$ ist ein Körper.
		
		Wesentliche Eigenschaften dieses Körpers.
		\begin{enumerate}
			\item Haben Abbildung $k \to P \to E_1 = P/m, a \mapsto \text{konst. Pol. $a$}$. Diese Abbildung ist injektiv, deshalb Inklusion von Körpern. Fasse ab sofort $k$ als Unterkörper von $E_1$ auf.
			\item Die Polynome $f \in \Lambda$ haben Nullstellen in $E_1$, nämlich $f(x_f) \in I \subset m$, also $f([x_f]) = 0$ in $E_1 = P / m$
			\item Die Körpererweiterung $E_1 / k$ ist algebraisch. Sei $a \in E_1$ irgendein Element. Schreibe $a = [g]$ wobei $g \in P$ ein Polynom in den endlich vielen Variablen $x_{\lambda_1}, \dots, x_{\lambda_n}$ ist.
			Dann $a \in k([x_{\lambda_1}], \dots, [x_{\lambda_n}]) \subset E_1$.
			
			Wir wissen aber für alle $i$ ist $[x_{\lambda_i}]$ Nullstelle des Polynoms $\lambda_i \in \Lambda$.
		\end{enumerate}
	
		\textit{Beobachtung:} Es ist nicht klar, dass $E_1$ ein algebraischer Abschluss von $k$ ist.
		
		\textit{Wissen:} Polnyome mit Koeffizienten in $k$ haben in $E_1$ Nullstelle.
		
		\textit{Wissen nicht:} Polynome mit Koeffizienten in $E_1$ haben in $E_1$ Nullstelle.
		
		Wiederhole diese Konstruktion, erhalte Erweiterungen
		\begin{equation*}
			k \subseteq E_1 \subseteq E_2 \subseteq \dots
		\end{equation*}
		so dass für alle $i \in \setN$ jedes nicht-konstante Polynom in $E_i[x]$ hat Nullstelle in $E_{i+1}$ und $E_{i+1} / E_i$ ist algebraisch. Insbesondere ist $E_i / k$ algebraisch.
		
		Setze
		\begin{equation*}
			E \coloneqq \bigcup_i E_i
		\end{equation*}
		dann gilt:
		\begin{enumerate}
			\item Weil ich eine Kette habe, ist $E$ ein Körper
			\item Gegeben $a \in E$. Dann $\exists x: a \in E_i$, also $a$ algebraisch über $k$. $\Rightarrow E/k$ ist algebraisch.
			\item Sei $f \in E[x]$ ein Polynom, $f(x) = \sum_{j=1}^n e_j x^j$. Dann gibt es ein $i \in \setN: \forall j: e_j \in E_i$. Also $f \in E_i[x]$ hat also eine Nullstelle in $E_{i+1} \subseteq E$.
		\end{enumerate}
	\end{proof}

	\begin{definition}
		Sei $k$ ein Körper, $f$ ein nicht konstantes Polynom, $f \in k[x]$. Eine Erweiterung $L/k$ heißt Zerfällungskörper von $f$, falls gilt:
		\begin{enumerate}
			\item $f$ zerfällt in $L[x]$ in Produkt von linearen Polynomen
			\begin{equation*}
				f = \operatorname{const} \cdot \prod (x - a_i) \in L[x]
			\end{equation*}
			\item $L = k(a_1, \dots, a_n)$
		\end{enumerate}
	\end{definition}

	\heading{Wesentliches Problem:} Gegeben $k$ und $f$, finde ein $L$.
	
	\begin{theorem}
		Sei $k$ ein Körper, dann gilt:
		\begin{enumerate}
			\item Jedes nicht-konstante $f$ hat einen Zerfällungskörper
			\item Gegeben $f$, dann sind je zwei Zerfällungskörper von $f$ isomorph
			\item Gegeben $f$ und ein Zerfällungskörper $L$, dann ist
			\begin{equation*}
				[L:k] \leq (\deg f)!
			\end{equation*}
		\end{enumerate}
	\end{theorem}

	\begin{proof}
		\heading{1)} Sei $f$ gegeben. Seien $a_1, \dots, a_n \in \overline{k}$ die Nullstellen, dann setze $L = k(a_1, \dots, a_n) \subseteq \overline{k}$
		
		\heading{2)} Sei $f$ gegeben. Wähle $L$ wie in Schritt 1), sei $L'$ ein weiterer Zerfällungskörper, seien $a_1', \dots, a_n'$ die Nullstellen von $f$ in $L'$.
		
		\textit{Wir wissen:} $L'/k$ ist algebraisch. Nach universeller Eigenschaft habe ich einen $k$-Morphismus
		\begin{equation*}
			\varphi: L' \to \overline{k} \supseteq L
		\end{equation*}
		
		\textit{Banale Beobachtung:} Die Abbildung $\varphi$ bildet Nullstellen von $f$ auf Nullstellen von $f$ in $\overline{k}$ ab. Sei $a_i \in L$ eine Nullstelle. Dann schreibe $f(x) = \sum f_i \cdot x^i$, wobei $f_i \in k$. Dann ist
		\begin{equation*}
			0_{\overline{k}} = \varphi(f(a)) = \varphi \left( \sum f_i \cdot a^i \right) = \sum \varphi(f_i) \cdot \varphi(a)^i = \sum f_i \varphi(a)^i = f(\varphi(a))
		\end{equation*}
		
		\textit{Also:} $\forall i: \varphi(a_i') = a_j$ für geeignetes $j$.
		
		\begin{equation*}
			\Rightarrow \operatorname{Bild}(\varphi) = \varphi(k(a_1', \dots, a_n')) \subseteq \underbrace{k(a_1, \dots, a_n)}_{=L} \subseteq \overline{k}
		\end{equation*}
		
		\textit{Andererseits:} $\operatorname{Bild}(\varphi)$ ist ein Zerfällungkörper, enthält alle $n$ Nullstellen $\Rightarrow \operatorname{Bild}(\varphi) = L$.
		
		$\Rightarrow \varphi$ ist isomorphismus
		
		\heading{3)} Sei $f$ gegeben, seien $a_1, \dots, a_n \in L$ die Nullstellen. Dann ist $L = k(a_1, \dots, a_n)$ und habe Kette.
		\begin{equation*}
			k \subseteq k(a_1) \subseteq k(a_1, a_2) \subseteq \dots
		\end{equation*}
		Dann:
		\begin{itemize}
			\item $f$ ist Polynom in $k$, das $a_1$ als Nullstelle hat
			\begin{equation*}
				[k(a_1) : k] \leq \deg f
			\end{equation*}
			\item $f/(x-a_1)$ ist Polynom in $k(a_1)$, das $a_2$ als Nullstelle hat
			\begin{equation*}
				[k(a_1, a_2) : k(a_1)] \leq n - 1
			\end{equation*}
			\item Wiederholte Anwendung:
			\begin{equation*}
				[L:k] \leq n!
			\end{equation*}
		\end{itemize}
		
	\end{proof}

	\begin{example}
		$k = \setQ$, $f = x^2 - 2$. Dann ist Zerfällungskörper
		\begin{equation*}
			L = \setQ(- \sqrt{2}, \sqrt{2})
		\end{equation*}
		und
		\begin{equation*}
			[\setQ(\sqrt{2}) : \setQ] = 2
		\end{equation*}
		und
		\begin{equation*}
			[\setQ(\sqrt{2}, - \sqrt{2}) : \setQ(\sqrt{2})] = 1
		\end{equation*}
		$\Rightarrow \deg[L : \setQ] = 2$.
	\end{example}

	\begin{example}
		$k = \setQ$, $f = x^3 - 2$. Dann:
		\begin{equation*}
			L = \setQ(\sqrt[3]{2}, \xi \sqrt[3]{2}, \xi^2 \sqrt[3]{2})
		\end{equation*}
		wobei $\xi = e^{\frac{2 \pi i}{3}}$, und
		\begin{equation*}
			[\setQ(\sqrt[3]{2}) : \setQ] = 3
		\end{equation*}
		und
		\begin{equation*}
			[\setQ(\xi \cdot \sqrt[3]{2}, \sqrt[3]{2}) : \setQ(\sqrt[3]{2})] = 2
		\end{equation*}
		weil $\setQ(\sqrt[3]{2}) \subset \setR$, $\xi \notin \setR$
		
		$\Rightarrow [L : \setQ] = 6$
	\end{example}

	\heading{Nächstes Ziel:} Zerfällungskörper verstehen. Dazu Nullstellenmenge von (irreduziblen) Polynomen verstehen.
	
	\heading{Dazu Sprache:} Sei $S \supseteq R$ eine Erweiterung von Ringen und sei $(a_\lambda)_{\lambda \in \Lambda}$ eine Familie von Elementen aus $S$ dann betrachte
	\begin{equation*}
		\bigcap_{\substack{\text{Zwischenringe} R \subseteq A \subseteq S \\ \forall \lambda \in \Lambda, a_\lambda \in A}} A = R[(a_\lambda)_{\lambda \in \Lambda}]
	\end{equation*}
	
	\heading{Fakt:}
	\begin{itemize}
		\item $R[(a_\lambda)_{\lambda \in \Lambda}]$ ist ein Unterring von $S$, der alle $a_\lambda$ enthält.
		\item $R[(a_\lambda)_{\lambda \in \Lambda}]$ ist der kleinste Unterring von $S$ der alle $(a_\lambda)_{\lambda \in \Lambda}$ enthält.
		\item Sei $\varphi: R[(x_\lambda)_{\lambda \in \Lambda}] \to S$ die eindeutige Abbildung, die $\forall \lambda$ $x_\lambda$ auf $a_\lambda$ abbildet. Dann ist $R[(a_\lambda)_{\lambda \in \Lambda}] = \operatorname{Bild}(\varphi)$
	\end{itemize}

	\heading{Auf Deutsch:} Elemente von $R[(a_\lambda)_{\lambda \in \Lambda}]$ sehen aus wie Polynome in $a_\lambda$.
	\begin{equation*}
		r_1 a_{\lambda_1}^7 a_{\lambda_2} + r_2 a_{\lambda_3}^8 \cdot a_{\lambda_4} \cdot a_{\lambda_1}
	\end{equation*}
	
	\heading{Spezialfall:} Die Ringe $R,S$ sind Körper. Gegeben also eine Körpererweiterung $L/k$ und Familie von Elementen aus $L$, $A \coloneqq (a_\lambda)_{\lambda \in \Lambda} \subseteq L$. Dann haben wir Ringe/Körper
	\begin{equation*}
		k \subseteq k[A] \overset{i}{\hookrightarrow} k(A) \subseteq L
	\end{equation*}
	und wir haben $k[A] \hookrightarrow Q(k[A])$. Und wir erhalten genau ein $\eta: Q(k[A]) \to k(A)$ wobei $\eta$ durch die universelle Eigenschaft des Quotientenkörpers gegeben ist.
	
	\heading{Klar:} $\operatorname{Bild}(\eta)$ ist Unterkörper von $k(A)$, der $k[A]$ enthält $\Rightarrow \operatorname{Bild}(\eta) = k(A)$. Also $\eta$ ist isomorph.
	
	\begin{theorem}
		Situation wie oben. Dann
		\begin{equation*}
			k(A) \cong Q(k(A))
		\end{equation*}
		mit kanonischer Isomorphie.
	\end{theorem}

	\begin{example}
		$k = \setR$, $L = \setC = \setR(i)$
		
		Wissen: jede komplexe Zahl kann ich schreiben als $r_1 + i r_2$, also $\setR[i] = \setR(i) = \setC$.
	\end{example}

	\heading{Allgemein:} Sei $L/k$ eine Körpererweiterung, sei $a \in L$ algebraisch über $k$. Dann kann ich alle Elemente von $k(a)$ schreiben als $k_0 + k_1 \cdot a + k_2 a^2 + \dots + k_{n-1} a^{n-1}$ wobei $n = [a : k]$. Also $k(a) = k[a]$.
	
	\begin{example}
		$L/k$ Körpererweiterung, $a \in L$ sei transzendent über $k$. Dann haben wir Abbildung
		\begin{equation*}
			k[x] \to k[a] \subseteq k(a), \quad f(x) \mapsto f(a)
		\end{equation*}
		per Definition ist $\varphi$ surjektiv. Per Annahme $a$ transzendent ist $\varphi$ injektiv. $\Rightarrow k[a] \cong k[x]$. Insbesonders $k[a]$ kein Körper also $\neq k(a)$. Induktiv beweist man:
	\end{example}

	\begin{theorem}
		Sei $L/k$ eine Körpererweiterung, seien $a_1, \dots, a_n \in L$ endlich viele Elemente, dann sind äquivalent
		\begin{enumerate}
			\item alle $a_i$ sind algebraisch
			\item $k[a_1, \dots, a_n] = k(a_1, \dots, a_n)$
		\end{enumerate}
	\end{theorem}

	\begin{remark}
		Achtung: für $\infty$ viele Elemente ist das Falsch! z.B. sei $L/k$ beliebig. $A = L$, dann ist $k[A] = k(A)$.
	\end{remark}

	\subsection{Separable und Inseparable Körpererweiterungen}
	
	\heading{Frage:} Sei $L/k$ Erweiterung, $a \in L$ sei algebraisch über $k$ und $f \in k[x]$ das Minimalpolynom. Kann $f$ mehrfache Nullstellen in $L$ haben?
	
	\heading{Teilantwort:} Wenn $k = \setQ$ ist, geht das nicht! Denn wenn $f$ die Zahl $a \in L$ als mehrfache Nullstelle hat, dann $f'(a) = 0$. $\lightning$ zur Annahme $f$ Minimalpolynom.
	
	\heading{Ziel:} Argument erweitern zu beliebigen Körpern
	
	\begin{definition}
		Sei $k$ ein Körper, $f \in k[x]$ ein Polynom. Dann schreibe
		\begin{equation*}
			f(x) = \sum_{i = 0}^n a_i \cdot x^i
		\end{equation*}
		und setze
		\begin{equation*}
			f'(x) = \sum_{i = 1}^n \underline{i} \cdot a_i \cdot x^{i-1}
		\end{equation*}
		wobei $\underline{i} = \underbrace{1 + \dots + 1}_{i-\text{mal}} \in k$
	\end{definition}

	\begin{theorem}
		Alle bekannten Ableitungsregeln gelten.
	\end{theorem}

	\heading{Zurück zur Frage:} Wenn $k$ ein beliebiger Körper der Charakteristik $0$ ist, und $a$ eine mehrfache Nullstelle von $f$ ist (d.h. in $L[x]$) kann ich schreiben
	\begin{equation*}
		f = (x-a)(x-a) \cdot \text{rest}
	\end{equation*}
	Dann sagt Ketten/Produkt-Regel dass $f'$ das $a$ immer noch als Nullstelle hat. Weil $\operatorname{char}(k) = 0$. $f' \not\equiv 0$. Also $\lightning$ wie oben.
	
	\begin{remark}
		In $\operatorname{char}(k) = p > 0$ immer noch wahr, dass $f'(a) = 0$ ist, aber es könnte sein, dass $f' \equiv 0$.
	\end{remark}

	\begin{definition}
		Eine irreduzibles Polynom $f$ heißt separabel, wenn $f$ in $\overline{k}$ keine mehrfache Nullstelle hat. Ein beliebiges Polynom $f$ ist separabel, wenn alle irreduziblen Faktoren separabel sind. Ansonsten nenne $f$ inseparabel.
	\end{definition}

	\begin{remark}
		Falls $\operatorname{char}(k) = 0$, sind alle Polynome separabel.
	\end{remark}

	\begin{remark}
		(Nicht-irreduzible) separable Polynome können mehrfache Nullstellen haben.
	\end{remark}

	\heading{Konstruktion mit Frobenius-Morphismus:} Sei $R$ ein Ring ein Ringmorphismus $R \to S$ induziert einen Ringmorphismus $R[x] \to S[x]$. Für $S = R$ und den Frobenius-Morphismus erhalten Ringmorphismus
	\begin{equation*}
		\eta: R[x] \to R[x], \quad \sum a_i x^i \mapsto a_i^p x^i
	\end{equation*}
	Falls $R$ Integritätsring ist, ist $\eta$ injektiv.
	
	$\operatorname{Bild}(\eta) = (R^p)[x] \subseteq R[x]$ und die Abbildung $\eta: R[x] \to R^p[x]$ ist Isomorphismus.
	
	\begin{theorem}[Charakterisierung inseparabler Polynome]
		Sei $k$ ein Körper, sei $f \in k[x]$ irreduzibel. Dann sind äquivalent
		\begin{enumerate}
			\item $f$ ist inseparabel
			\item $f' \equiv 0$
			\item $p = \operatorname{char}(k)$ ist eine Primzahl. Es gibt ein irreduzibles separables $g \in k[x]$ und $n \in \setN$ so dass $f(x) = g(x^{p \cdot n}) = g((x^p)^n)$.
		\end{enumerate}
	\end{theorem}

	\begin{proof}
		\heading{$1) \Rightarrow 2)$:} Sei $f$ inseparabel, d.h. $f$ hat in $\overline{k}$ eine mehrfache Nullstelle $a$, dann ist auch $f'(a) = 0$. Widerspruch zur Irreduzibilität falls $f \not\equiv 0$. Also $f' \equiv 0$.
		
		\heading{$2) \Rightarrow 3)$:} Sei $f(x) = \sum_{i=0}^n a_i x^i$. Dann:
		\begin{equation*}
			f'(x) = \sum_{i = 1}^n a_i \cdot i \cdot x^{i-1}
		\end{equation*}
		wobei $i$ hier $\varphi(i) = \underbrace{1 + \dots + 1}_\text{$i$-mal}$.
		
		Falls $\operatorname{char}(k) = 0$ wäre dann ist $\forall i$ mit $a_i \neq 0$ auch $i \cdot a_i \neq 0$ also $f'(x) \not\equiv 0$ $\lightning$. Also ist $\operatorname{char}(k) = p > 0$. Die Zahl $p$ ist prim weil $k$ ein Körper ist.
		
		\textit{Beobachtung:} Falls $i$ kein Vielfaches von $p$ ist, dann $\varphi(i) \neq 0$. Es ist aber $a_i \cdot \varphi(i) = 0 \Rightarrow a_i = 0$ für alle $i$ die kein Vielfaches von $p$ sind. Also
		\begin{equation*}
			f(x) = \sum_{j = 0}^{n/p} a_{j \cdot p} x^{j \cdot p}
		\end{equation*}
		Setze $g_1(x) = \sum_{j = 0}^{n/p} a_{j \cdot p} x^j$. Dann $f(x) = g_1(x^p)$.
		
		\textit{Idee:} Falls $g_1$ inseparabel ist, wiederhole Prozedur, finde $g_2(x)$ so dass $g_1(x) = g_2(x^p)$ ($\Rightarrow f(x) = g_2(x^{2p})$). Weil der Grad der Polynome dabei sinkt terminiert diese Prozedur nach endlich vielen Schritten, finde $g = g_n$ so dass $f(x) = g(x^{n \cdot p})$ und $g$ separabel ist.
		
		Damit das funktioniert muss ich zeigen, dass $g_1$ irreduzibel ist (per Induktion sind dann auch $g_2, \dots, g_n = g$ irreduzibel).
		
		\textit{Erinnerung:} hatten Morphismen
		\begin{equation*}
			\varphi_1: k[x] \to (k^p)[x], \quad \sum h_i \cdot x^i \mapsto \sum h_i^p \cdot x_i
		\end{equation*}
		
		\begin{equation*}
			\mathcal{F}: k[x] \to (k^p)[x^p] \subseteq (k^n)[x] \subseteq k[x], \quad \sum h_i \cdot x^i \mapsto \sum h_i^p \cdot x^{i \cdot p}
		\end{equation*}
		
		Nachrechnen: es ist $\varphi(f) \in (k^p)[x^p]$ weil $f \in k[x^p]$ und $g = \mathcal{F}^{-1}(\varphi(f))$. Da $\varphi, \mathcal{F}$ Isomorphismen sind folgt aus $f$ irreduzibel $g_1$ irreduzibel.
		
		\heading{$3) \Rightarrow 1)$:} Angenommen $f$ hat folgende Eigenschaft: $\exists g(x) \in k[x]: f(x) = g(x^p)$. Sei $a \in \overline{k}$ eine Nullstelle von $g$, d.h. $g(x) = (x - a) \cdot \text{rest}$ in $\overline{k}[x]$. Wähle $b \in \overline{k}$ mit $b^p = a$ (das geht, weil $\overline{k}$ algebraisch abgeschlossen ist). Dann
		\begin{equation*}
			g(x^p) = (x^p - b^p) \cdot \text{rest} = (x - b)^p \cdot \text{rest}
		\end{equation*}
		$\Rightarrow b \in \overline{k}$ ist $p$-fache Nullstelle von $f$, also $f$ inseparabel.
	\end{proof}

	Warum diese Diskussion von Inseparabilität? Antwort kommt jetzt!
	
	\begin{lemma}
		Sei $L/k$ eine Körpererweiterung und $a \in L$, sei algebraisch über $k$. Setze $M = k(a)$. Sei $f(x) \in k[x]$ das Minimalpolynom von $a$. Angenommen $f$ hat exakt $m$ unterschiedliche Nullstellen in $\overline{k}$. Dann gibt es genau $m$ unterschiedliche $k$-Morphismen
		\begin{equation*}
			\varphi: M \to \overline{k}
		\end{equation*}
	\end{lemma}

	\begin{remark}
		Falls $f$ separabel ist, $m = \deg f$. Falls $f$ inseparabel ist, ist $m < \deg f$.
	\end{remark}

	\begin{proof}
		\textit{Beobachtung 1:} Wir wissen schon: Die Elemente von $M$ kann ich schreiben als
		\begin{equation*}
			\lambda_0 ü \lambda_1 a + \lambda_2 a^2 + \dots + \lambda_{n-1} a^{n-1}
		\end{equation*}
		mit $\lambda_i \in k$ wobei $n = \deg f$. Insbesondere ist für alle solche Elemente
		\begin{equation*}
			\varphi(\lambda_0 + \lambda_1 a + \dots + \lambda_{n-1} a^{n-1}) = \sum \lambda_i \varphi(a)^i
		\end{equation*}
		das bedeutet: $\varphi$ ist durch $\varphi(a)$ eindeutig festgelegt!
		
		\textit{Beobachtung 2:} Gegeben einen $k$-Morphismus $\varphi$, dann ist $\varphi(a)$ eine Nullstelle des Polynoms $f(x) \in k[x]$ habe aber nur $m$ unterschiedliche Nullstellen!
		
		Insgesamt also höchstens $m$ unterschiedliche Morphismen!
		
		Noch zu zeigen: Wenn $b \in \overline{k}$ eine Nullstelle von $f$ ist, dann existiert ein $k$-Morphismus $\varphi: M \to \overline{k}$ so dass $\varphi(a) = b$ ist.
		
		\textit{Erinnerung:} Wir wissen $M \simeq k[x]/(f)$ wobei $a$ mit $[x]$ identifiziert wird.
		
		Haben Morphismus:
		\begin{equation*}
			\Omega: k[x] \to \overline{k}, \quad g \mapsto g(b)
		\end{equation*}
		Dann $f \in \operatorname{Ker}(\Omega)$, der Kern ist ein Hauptideal und $f$ reduzibel also: $(f) = \operatorname{ker}(\Omega)$. Also erhalte (nach universeller Eigenschaft) einen Morphismus $k[x]/(f) \to \overline{k}$ wobei $[x] \mapsto b$
		
		Erhalte $M \to \overline{k}$ durch Komposition der Morphismen.
		
		\heading{Varianten mit völlig analogem Beweis}
		
		\begin{lemma}
			Sei $L/k$ eine Körpererweiterung. $a \in L$ algebraisch mit Minimalpolynom $f \in k[x]$. $f$ hat $m$ unterschiedliche Nullstellen in $L$. Dann gibt es genau $m$ unterschiedliche $k$-Morphismen $\varphi: M \to L$, wobei $M = L(a)$ ist.
		\end{lemma}
	
		\begin{lemma}
			Seien $L_1$ und $L_2$ Körper und $\sigma: L_1 \to L_2$ Körpermorphismen. $a \in L_2$ sei algebraisch über $\operatorname{Bild}(\sigma)$ mit Minimalpolynom $f$. Angenommen $f$ hat $m$ unterschiedliche Nullstellen in $L_2$. Dann gibt es genau $m$ unterschiedliche Fortsetzungen von $\sigma$ zu Morphismen $\Sigma: M \to L_2$, wobei $M \supseteq L$, der Körper $\sigma(L_1)(a)$.
		\end{lemma}
	\end{proof}
	
\end{document}